
% Default to the notebook output style

    


% Inherit from the specified cell style.




    
\documentclass[11pt]{article}

    
    
    \usepackage[T1]{fontenc}
    % Nicer default font (+ math font) than Computer Modern for most use cases
    \usepackage{mathpazo}

    % Basic figure setup, for now with no caption control since it's done
    % automatically by Pandoc (which extracts ![](path) syntax from Markdown).
    \usepackage{graphicx}
    % We will generate all images so they have a width \maxwidth. This means
    % that they will get their normal width if they fit onto the page, but
    % are scaled down if they would overflow the margins.
    \makeatletter
    \def\maxwidth{\ifdim\Gin@nat@width>\linewidth\linewidth
    \else\Gin@nat@width\fi}
    \makeatother
    \let\Oldincludegraphics\includegraphics
    % Set max figure width to be 80% of text width, for now hardcoded.
    \renewcommand{\includegraphics}[1]{\Oldincludegraphics[width=.8\maxwidth]{#1}}
    % Ensure that by default, figures have no caption (until we provide a
    % proper Figure object with a Caption API and a way to capture that
    % in the conversion process - todo).
    \usepackage{caption}
    \DeclareCaptionLabelFormat{nolabel}{}
    \captionsetup{labelformat=nolabel}

    \usepackage{adjustbox} % Used to constrain images to a maximum size 
    \usepackage{xcolor} % Allow colors to be defined
    \usepackage{enumerate} % Needed for markdown enumerations to work
    \usepackage{geometry} % Used to adjust the document margins
    \usepackage{amsmath} % Equations
    \usepackage{amssymb} % Equations
    \usepackage{textcomp} % defines textquotesingle
    % Hack from http://tex.stackexchange.com/a/47451/13684:
    \AtBeginDocument{%
        \def\PYZsq{\textquotesingle}% Upright quotes in Pygmentized code
    }
    \usepackage{upquote} % Upright quotes for verbatim code
    \usepackage{eurosym} % defines \euro
    \usepackage[mathletters]{ucs} % Extended unicode (utf-8) support
    \usepackage[utf8x]{inputenc} % Allow utf-8 characters in the tex document
    \usepackage{fancyvrb} % verbatim replacement that allows latex
    \usepackage{grffile} % extends the file name processing of package graphics 
                         % to support a larger range 
    % The hyperref package gives us a pdf with properly built
    % internal navigation ('pdf bookmarks' for the table of contents,
    % internal cross-reference links, web links for URLs, etc.)
    \usepackage{hyperref}
    \usepackage{longtable} % longtable support required by pandoc >1.10
    \usepackage{booktabs}  % table support for pandoc > 1.12.2
    \usepackage[inline]{enumitem} % IRkernel/repr support (it uses the enumerate* environment)
    \usepackage[normalem]{ulem} % ulem is needed to support strikethroughs (\sout)
                                % normalem makes italics be italics, not underlines
    

    
    
    % Colors for the hyperref package
    \definecolor{urlcolor}{rgb}{0,.145,.698}
    \definecolor{linkcolor}{rgb}{.71,0.21,0.01}
    \definecolor{citecolor}{rgb}{.12,.54,.11}

    % ANSI colors
    \definecolor{ansi-black}{HTML}{3E424D}
    \definecolor{ansi-black-intense}{HTML}{282C36}
    \definecolor{ansi-red}{HTML}{E75C58}
    \definecolor{ansi-red-intense}{HTML}{B22B31}
    \definecolor{ansi-green}{HTML}{00A250}
    \definecolor{ansi-green-intense}{HTML}{007427}
    \definecolor{ansi-yellow}{HTML}{DDB62B}
    \definecolor{ansi-yellow-intense}{HTML}{B27D12}
    \definecolor{ansi-blue}{HTML}{208FFB}
    \definecolor{ansi-blue-intense}{HTML}{0065CA}
    \definecolor{ansi-magenta}{HTML}{D160C4}
    \definecolor{ansi-magenta-intense}{HTML}{A03196}
    \definecolor{ansi-cyan}{HTML}{60C6C8}
    \definecolor{ansi-cyan-intense}{HTML}{258F8F}
    \definecolor{ansi-white}{HTML}{C5C1B4}
    \definecolor{ansi-white-intense}{HTML}{A1A6B2}

    % commands and environments needed by pandoc snippets
    % extracted from the output of `pandoc -s`
    \providecommand{\tightlist}{%
      \setlength{\itemsep}{0pt}\setlength{\parskip}{0pt}}
    \DefineVerbatimEnvironment{Highlighting}{Verbatim}{commandchars=\\\{\}}
    % Add ',fontsize=\small' for more characters per line
    \newenvironment{Shaded}{}{}
    \newcommand{\KeywordTok}[1]{\textcolor[rgb]{0.00,0.44,0.13}{\textbf{{#1}}}}
    \newcommand{\DataTypeTok}[1]{\textcolor[rgb]{0.56,0.13,0.00}{{#1}}}
    \newcommand{\DecValTok}[1]{\textcolor[rgb]{0.25,0.63,0.44}{{#1}}}
    \newcommand{\BaseNTok}[1]{\textcolor[rgb]{0.25,0.63,0.44}{{#1}}}
    \newcommand{\FloatTok}[1]{\textcolor[rgb]{0.25,0.63,0.44}{{#1}}}
    \newcommand{\CharTok}[1]{\textcolor[rgb]{0.25,0.44,0.63}{{#1}}}
    \newcommand{\StringTok}[1]{\textcolor[rgb]{0.25,0.44,0.63}{{#1}}}
    \newcommand{\CommentTok}[1]{\textcolor[rgb]{0.38,0.63,0.69}{\textit{{#1}}}}
    \newcommand{\OtherTok}[1]{\textcolor[rgb]{0.00,0.44,0.13}{{#1}}}
    \newcommand{\AlertTok}[1]{\textcolor[rgb]{1.00,0.00,0.00}{\textbf{{#1}}}}
    \newcommand{\FunctionTok}[1]{\textcolor[rgb]{0.02,0.16,0.49}{{#1}}}
    \newcommand{\RegionMarkerTok}[1]{{#1}}
    \newcommand{\ErrorTok}[1]{\textcolor[rgb]{1.00,0.00,0.00}{\textbf{{#1}}}}
    \newcommand{\NormalTok}[1]{{#1}}
    
    % Additional commands for more recent versions of Pandoc
    \newcommand{\ConstantTok}[1]{\textcolor[rgb]{0.53,0.00,0.00}{{#1}}}
    \newcommand{\SpecialCharTok}[1]{\textcolor[rgb]{0.25,0.44,0.63}{{#1}}}
    \newcommand{\VerbatimStringTok}[1]{\textcolor[rgb]{0.25,0.44,0.63}{{#1}}}
    \newcommand{\SpecialStringTok}[1]{\textcolor[rgb]{0.73,0.40,0.53}{{#1}}}
    \newcommand{\ImportTok}[1]{{#1}}
    \newcommand{\DocumentationTok}[1]{\textcolor[rgb]{0.73,0.13,0.13}{\textit{{#1}}}}
    \newcommand{\AnnotationTok}[1]{\textcolor[rgb]{0.38,0.63,0.69}{\textbf{\textit{{#1}}}}}
    \newcommand{\CommentVarTok}[1]{\textcolor[rgb]{0.38,0.63,0.69}{\textbf{\textit{{#1}}}}}
    \newcommand{\VariableTok}[1]{\textcolor[rgb]{0.10,0.09,0.49}{{#1}}}
    \newcommand{\ControlFlowTok}[1]{\textcolor[rgb]{0.00,0.44,0.13}{\textbf{{#1}}}}
    \newcommand{\OperatorTok}[1]{\textcolor[rgb]{0.40,0.40,0.40}{{#1}}}
    \newcommand{\BuiltInTok}[1]{{#1}}
    \newcommand{\ExtensionTok}[1]{{#1}}
    \newcommand{\PreprocessorTok}[1]{\textcolor[rgb]{0.74,0.48,0.00}{{#1}}}
    \newcommand{\AttributeTok}[1]{\textcolor[rgb]{0.49,0.56,0.16}{{#1}}}
    \newcommand{\InformationTok}[1]{\textcolor[rgb]{0.38,0.63,0.69}{\textbf{\textit{{#1}}}}}
    \newcommand{\WarningTok}[1]{\textcolor[rgb]{0.38,0.63,0.69}{\textbf{\textit{{#1}}}}}
    
    
    % Define a nice break command that doesn't care if a line doesn't already
    % exist.
    \def\br{\hspace*{\fill} \\* }
    % Math Jax compatability definitions
    \def\gt{>}
    \def\lt{<}
    % Document parameters
    \title{Time Series Analysis}
    
    
    

    % Pygments definitions
    
\makeatletter
\def\PY@reset{\let\PY@it=\relax \let\PY@bf=\relax%
    \let\PY@ul=\relax \let\PY@tc=\relax%
    \let\PY@bc=\relax \let\PY@ff=\relax}
\def\PY@tok#1{\csname PY@tok@#1\endcsname}
\def\PY@toks#1+{\ifx\relax#1\empty\else%
    \PY@tok{#1}\expandafter\PY@toks\fi}
\def\PY@do#1{\PY@bc{\PY@tc{\PY@ul{%
    \PY@it{\PY@bf{\PY@ff{#1}}}}}}}
\def\PY#1#2{\PY@reset\PY@toks#1+\relax+\PY@do{#2}}

\expandafter\def\csname PY@tok@w\endcsname{\def\PY@tc##1{\textcolor[rgb]{0.73,0.73,0.73}{##1}}}
\expandafter\def\csname PY@tok@c\endcsname{\let\PY@it=\textit\def\PY@tc##1{\textcolor[rgb]{0.25,0.50,0.50}{##1}}}
\expandafter\def\csname PY@tok@cp\endcsname{\def\PY@tc##1{\textcolor[rgb]{0.74,0.48,0.00}{##1}}}
\expandafter\def\csname PY@tok@k\endcsname{\let\PY@bf=\textbf\def\PY@tc##1{\textcolor[rgb]{0.00,0.50,0.00}{##1}}}
\expandafter\def\csname PY@tok@kp\endcsname{\def\PY@tc##1{\textcolor[rgb]{0.00,0.50,0.00}{##1}}}
\expandafter\def\csname PY@tok@kt\endcsname{\def\PY@tc##1{\textcolor[rgb]{0.69,0.00,0.25}{##1}}}
\expandafter\def\csname PY@tok@o\endcsname{\def\PY@tc##1{\textcolor[rgb]{0.40,0.40,0.40}{##1}}}
\expandafter\def\csname PY@tok@ow\endcsname{\let\PY@bf=\textbf\def\PY@tc##1{\textcolor[rgb]{0.67,0.13,1.00}{##1}}}
\expandafter\def\csname PY@tok@nb\endcsname{\def\PY@tc##1{\textcolor[rgb]{0.00,0.50,0.00}{##1}}}
\expandafter\def\csname PY@tok@nf\endcsname{\def\PY@tc##1{\textcolor[rgb]{0.00,0.00,1.00}{##1}}}
\expandafter\def\csname PY@tok@nc\endcsname{\let\PY@bf=\textbf\def\PY@tc##1{\textcolor[rgb]{0.00,0.00,1.00}{##1}}}
\expandafter\def\csname PY@tok@nn\endcsname{\let\PY@bf=\textbf\def\PY@tc##1{\textcolor[rgb]{0.00,0.00,1.00}{##1}}}
\expandafter\def\csname PY@tok@ne\endcsname{\let\PY@bf=\textbf\def\PY@tc##1{\textcolor[rgb]{0.82,0.25,0.23}{##1}}}
\expandafter\def\csname PY@tok@nv\endcsname{\def\PY@tc##1{\textcolor[rgb]{0.10,0.09,0.49}{##1}}}
\expandafter\def\csname PY@tok@no\endcsname{\def\PY@tc##1{\textcolor[rgb]{0.53,0.00,0.00}{##1}}}
\expandafter\def\csname PY@tok@nl\endcsname{\def\PY@tc##1{\textcolor[rgb]{0.63,0.63,0.00}{##1}}}
\expandafter\def\csname PY@tok@ni\endcsname{\let\PY@bf=\textbf\def\PY@tc##1{\textcolor[rgb]{0.60,0.60,0.60}{##1}}}
\expandafter\def\csname PY@tok@na\endcsname{\def\PY@tc##1{\textcolor[rgb]{0.49,0.56,0.16}{##1}}}
\expandafter\def\csname PY@tok@nt\endcsname{\let\PY@bf=\textbf\def\PY@tc##1{\textcolor[rgb]{0.00,0.50,0.00}{##1}}}
\expandafter\def\csname PY@tok@nd\endcsname{\def\PY@tc##1{\textcolor[rgb]{0.67,0.13,1.00}{##1}}}
\expandafter\def\csname PY@tok@s\endcsname{\def\PY@tc##1{\textcolor[rgb]{0.73,0.13,0.13}{##1}}}
\expandafter\def\csname PY@tok@sd\endcsname{\let\PY@it=\textit\def\PY@tc##1{\textcolor[rgb]{0.73,0.13,0.13}{##1}}}
\expandafter\def\csname PY@tok@si\endcsname{\let\PY@bf=\textbf\def\PY@tc##1{\textcolor[rgb]{0.73,0.40,0.53}{##1}}}
\expandafter\def\csname PY@tok@se\endcsname{\let\PY@bf=\textbf\def\PY@tc##1{\textcolor[rgb]{0.73,0.40,0.13}{##1}}}
\expandafter\def\csname PY@tok@sr\endcsname{\def\PY@tc##1{\textcolor[rgb]{0.73,0.40,0.53}{##1}}}
\expandafter\def\csname PY@tok@ss\endcsname{\def\PY@tc##1{\textcolor[rgb]{0.10,0.09,0.49}{##1}}}
\expandafter\def\csname PY@tok@sx\endcsname{\def\PY@tc##1{\textcolor[rgb]{0.00,0.50,0.00}{##1}}}
\expandafter\def\csname PY@tok@m\endcsname{\def\PY@tc##1{\textcolor[rgb]{0.40,0.40,0.40}{##1}}}
\expandafter\def\csname PY@tok@gh\endcsname{\let\PY@bf=\textbf\def\PY@tc##1{\textcolor[rgb]{0.00,0.00,0.50}{##1}}}
\expandafter\def\csname PY@tok@gu\endcsname{\let\PY@bf=\textbf\def\PY@tc##1{\textcolor[rgb]{0.50,0.00,0.50}{##1}}}
\expandafter\def\csname PY@tok@gd\endcsname{\def\PY@tc##1{\textcolor[rgb]{0.63,0.00,0.00}{##1}}}
\expandafter\def\csname PY@tok@gi\endcsname{\def\PY@tc##1{\textcolor[rgb]{0.00,0.63,0.00}{##1}}}
\expandafter\def\csname PY@tok@gr\endcsname{\def\PY@tc##1{\textcolor[rgb]{1.00,0.00,0.00}{##1}}}
\expandafter\def\csname PY@tok@ge\endcsname{\let\PY@it=\textit}
\expandafter\def\csname PY@tok@gs\endcsname{\let\PY@bf=\textbf}
\expandafter\def\csname PY@tok@gp\endcsname{\let\PY@bf=\textbf\def\PY@tc##1{\textcolor[rgb]{0.00,0.00,0.50}{##1}}}
\expandafter\def\csname PY@tok@go\endcsname{\def\PY@tc##1{\textcolor[rgb]{0.53,0.53,0.53}{##1}}}
\expandafter\def\csname PY@tok@gt\endcsname{\def\PY@tc##1{\textcolor[rgb]{0.00,0.27,0.87}{##1}}}
\expandafter\def\csname PY@tok@err\endcsname{\def\PY@bc##1{\setlength{\fboxsep}{0pt}\fcolorbox[rgb]{1.00,0.00,0.00}{1,1,1}{\strut ##1}}}
\expandafter\def\csname PY@tok@kc\endcsname{\let\PY@bf=\textbf\def\PY@tc##1{\textcolor[rgb]{0.00,0.50,0.00}{##1}}}
\expandafter\def\csname PY@tok@kd\endcsname{\let\PY@bf=\textbf\def\PY@tc##1{\textcolor[rgb]{0.00,0.50,0.00}{##1}}}
\expandafter\def\csname PY@tok@kn\endcsname{\let\PY@bf=\textbf\def\PY@tc##1{\textcolor[rgb]{0.00,0.50,0.00}{##1}}}
\expandafter\def\csname PY@tok@kr\endcsname{\let\PY@bf=\textbf\def\PY@tc##1{\textcolor[rgb]{0.00,0.50,0.00}{##1}}}
\expandafter\def\csname PY@tok@bp\endcsname{\def\PY@tc##1{\textcolor[rgb]{0.00,0.50,0.00}{##1}}}
\expandafter\def\csname PY@tok@fm\endcsname{\def\PY@tc##1{\textcolor[rgb]{0.00,0.00,1.00}{##1}}}
\expandafter\def\csname PY@tok@vc\endcsname{\def\PY@tc##1{\textcolor[rgb]{0.10,0.09,0.49}{##1}}}
\expandafter\def\csname PY@tok@vg\endcsname{\def\PY@tc##1{\textcolor[rgb]{0.10,0.09,0.49}{##1}}}
\expandafter\def\csname PY@tok@vi\endcsname{\def\PY@tc##1{\textcolor[rgb]{0.10,0.09,0.49}{##1}}}
\expandafter\def\csname PY@tok@vm\endcsname{\def\PY@tc##1{\textcolor[rgb]{0.10,0.09,0.49}{##1}}}
\expandafter\def\csname PY@tok@sa\endcsname{\def\PY@tc##1{\textcolor[rgb]{0.73,0.13,0.13}{##1}}}
\expandafter\def\csname PY@tok@sb\endcsname{\def\PY@tc##1{\textcolor[rgb]{0.73,0.13,0.13}{##1}}}
\expandafter\def\csname PY@tok@sc\endcsname{\def\PY@tc##1{\textcolor[rgb]{0.73,0.13,0.13}{##1}}}
\expandafter\def\csname PY@tok@dl\endcsname{\def\PY@tc##1{\textcolor[rgb]{0.73,0.13,0.13}{##1}}}
\expandafter\def\csname PY@tok@s2\endcsname{\def\PY@tc##1{\textcolor[rgb]{0.73,0.13,0.13}{##1}}}
\expandafter\def\csname PY@tok@sh\endcsname{\def\PY@tc##1{\textcolor[rgb]{0.73,0.13,0.13}{##1}}}
\expandafter\def\csname PY@tok@s1\endcsname{\def\PY@tc##1{\textcolor[rgb]{0.73,0.13,0.13}{##1}}}
\expandafter\def\csname PY@tok@mb\endcsname{\def\PY@tc##1{\textcolor[rgb]{0.40,0.40,0.40}{##1}}}
\expandafter\def\csname PY@tok@mf\endcsname{\def\PY@tc##1{\textcolor[rgb]{0.40,0.40,0.40}{##1}}}
\expandafter\def\csname PY@tok@mh\endcsname{\def\PY@tc##1{\textcolor[rgb]{0.40,0.40,0.40}{##1}}}
\expandafter\def\csname PY@tok@mi\endcsname{\def\PY@tc##1{\textcolor[rgb]{0.40,0.40,0.40}{##1}}}
\expandafter\def\csname PY@tok@il\endcsname{\def\PY@tc##1{\textcolor[rgb]{0.40,0.40,0.40}{##1}}}
\expandafter\def\csname PY@tok@mo\endcsname{\def\PY@tc##1{\textcolor[rgb]{0.40,0.40,0.40}{##1}}}
\expandafter\def\csname PY@tok@ch\endcsname{\let\PY@it=\textit\def\PY@tc##1{\textcolor[rgb]{0.25,0.50,0.50}{##1}}}
\expandafter\def\csname PY@tok@cm\endcsname{\let\PY@it=\textit\def\PY@tc##1{\textcolor[rgb]{0.25,0.50,0.50}{##1}}}
\expandafter\def\csname PY@tok@cpf\endcsname{\let\PY@it=\textit\def\PY@tc##1{\textcolor[rgb]{0.25,0.50,0.50}{##1}}}
\expandafter\def\csname PY@tok@c1\endcsname{\let\PY@it=\textit\def\PY@tc##1{\textcolor[rgb]{0.25,0.50,0.50}{##1}}}
\expandafter\def\csname PY@tok@cs\endcsname{\let\PY@it=\textit\def\PY@tc##1{\textcolor[rgb]{0.25,0.50,0.50}{##1}}}

\def\PYZbs{\char`\\}
\def\PYZus{\char`\_}
\def\PYZob{\char`\{}
\def\PYZcb{\char`\}}
\def\PYZca{\char`\^}
\def\PYZam{\char`\&}
\def\PYZlt{\char`\<}
\def\PYZgt{\char`\>}
\def\PYZsh{\char`\#}
\def\PYZpc{\char`\%}
\def\PYZdl{\char`\$}
\def\PYZhy{\char`\-}
\def\PYZsq{\char`\'}
\def\PYZdq{\char`\"}
\def\PYZti{\char`\~}
% for compatibility with earlier versions
\def\PYZat{@}
\def\PYZlb{[}
\def\PYZrb{]}
\makeatother


    % Exact colors from NB
    \definecolor{incolor}{rgb}{0.0, 0.0, 0.5}
    \definecolor{outcolor}{rgb}{0.545, 0.0, 0.0}



    
    % Prevent overflowing lines due to hard-to-break entities
    \sloppy 
    % Setup hyperref package
    \hypersetup{
      breaklinks=true,  % so long urls are correctly broken across lines
      colorlinks=true,
      urlcolor=urlcolor,
      linkcolor=linkcolor,
      citecolor=citecolor,
      }
    % Slightly bigger margins than the latex defaults
    
    \geometry{verbose,tmargin=1in,bmargin=1in,lmargin=1in,rmargin=1in}
    
    

    \begin{document}
    
    
    \maketitle
    
    

    
    \section{BT2101 Introduction to Time
Series}\label{bt2101-introduction-to-time-series}

    \subsection{1 Goal}\label{goal}

In this notebook, we will explore \textbf{Time Series} including: *
Understand non-stationary and stationary time series * Indentifying
trend, seasonality, residual and autocorrelation * Run and fit ARMA,
ARIMA model * Do Forecasting

For the \textbf{Time Series} method, you will: * Use open-source package
\texttt{pandas} and \texttt{statsmodels} to do time series analysis *
Understand how to transform the non-stationary time series data into
stationary one, and then fit with ARMA model * References:
\href{https://www.apress.com/gp/book/9781484228227}{Haroon, D. Python
Machine Learning Case Studies. Apress..}

    \begin{Verbatim}[commandchars=\\\{\}]
{\color{incolor}In [{\color{incolor}1}]:} \PY{c+c1}{\PYZsh{} \PYZhy{}*\PYZhy{} coding:utf\PYZhy{}8 \PYZhy{}*\PYZhy{}}
        \PY{k+kn}{import} \PY{n+nn}{numpy} \PY{k}{as} \PY{n+nn}{np}
        \PY{k+kn}{import} \PY{n+nn}{pandas} \PY{k}{as} \PY{n+nn}{pd}
        \PY{k+kn}{import} \PY{n+nn}{matplotlib}\PY{n+nn}{.}\PY{n+nn}{pyplot} \PY{k}{as} \PY{n+nn}{plt}
        \PY{k+kn}{from} \PY{n+nn}{math} \PY{k}{import} \PY{n}{sqrt}
        \PY{k+kn}{import} \PY{n+nn}{statsmodels}\PY{n+nn}{.}\PY{n+nn}{api} \PY{k}{as} \PY{n+nn}{sm}
        \PY{k+kn}{from} \PY{n+nn}{IPython}\PY{n+nn}{.}\PY{n+nn}{display} \PY{k}{import} \PY{n}{Image}
        \PY{k+kn}{from} \PY{n+nn}{matplotlib}\PY{n+nn}{.}\PY{n+nn}{pylab} \PY{k}{import} \PY{n}{rcParams}
        \PY{k+kn}{from} \PY{n+nn}{statsmodels}\PY{n+nn}{.}\PY{n+nn}{tsa}\PY{n+nn}{.}\PY{n+nn}{stattools} \PY{k}{import} \PY{n}{adfuller}
        \PY{k+kn}{from} \PY{n+nn}{statsmodels}\PY{n+nn}{.}\PY{n+nn}{tsa}\PY{n+nn}{.}\PY{n+nn}{stattools} \PY{k}{import} \PY{n}{acf}\PY{p}{,} \PY{n}{pacf}
        \PY{k+kn}{from} \PY{n+nn}{sklearn}\PY{n+nn}{.}\PY{n+nn}{linear\PYZus{}model} \PY{k}{import} \PY{n}{LinearRegression}
        \PY{k+kn}{from} \PY{n+nn}{statsmodels}\PY{n+nn}{.}\PY{n+nn}{tsa}\PY{n+nn}{.}\PY{n+nn}{arima\PYZus{}model} \PY{k}{import} \PY{n}{ARMA}\PY{p}{,} \PY{n}{ARIMA}
        \PY{k+kn}{from} \PY{n+nn}{sklearn}\PY{n+nn}{.}\PY{n+nn}{metrics} \PY{k}{import} \PY{n}{explained\PYZus{}variance\PYZus{}score}
        \PY{k+kn}{from} \PY{n+nn}{\PYZus{}\PYZus{}future\PYZus{}\PYZus{}} \PY{k}{import} \PY{n}{division}\PY{p}{,} \PY{n}{print\PYZus{}function}
        \PY{k+kn}{from} \PY{n+nn}{datetime} \PY{k}{import} \PY{n}{datetime}
        \PY{o}{\PYZpc{}}\PY{k}{matplotlib} inline
\end{Verbatim}


    \subsection{2 Time Series Analysis}\label{time-series-analysis}

    \subsubsection{2.1 Stationary and Non-stationary Time
Series}\label{stationary-and-non-stationary-time-series}

A time series data \({x_t}, t=0,...\) is \textbf{weakly stationary} as
long as: 1. \(E(x_t)\) is constant 2. \(VAR(x_t)\) is constant 3.
\(COV(x_t, x_{t+k})\) is independent of t

\textbf{Why do we care about `stationarity' of a time series?}

The reason we took up this section first was that until unless our time
series is stationary, we cannot build a time series model. In cases
where the stationary criterion are violated, the first requisite becomes
to stationarize the time series and then try stochastic models to
predict this time series. There are multiple ways of bringing this
stationarity. Some of them are Detrending, Differencing etc.
https://www.analyticsvidhya.com/blog/2015/12/complete-tutorial-time-series-modeling/

    \subsubsection{2.2 Stationarize Time
Series}\label{stationarize-time-series}

\paragraph{An additive model suggests that a time series is composed
by}\label{an-additive-model-suggests-that-a-time-series-is-composed-by}

\begin{itemize}
\tightlist
\item
  Level
\item
  Trend
\item
  Seasonality
\item
  Random noise 
\end{itemize}

\paragraph{De-trending or Rolling}\label{de-trending-or-rolling}

Remove the trend component from the time series. For example, * model
the trend on time series: \(x_t=level+trend\times t+error\) * or mean
aggregation: For each time stamp, calculate mean value within a neighbor
time window

And then remove the trending component and further build models on the
rest.

\paragraph{Differencing}\label{differencing}

Usually we can use AR(I)MA model to remove non-stationarity and fit the
time series data. Please refer to the lecture slides.

\paragraph{A typical procedure of time series
forecasting}\label{a-typical-procedure-of-time-series-forecasting}

\begin{figure}
\centering
\includegraphics{procedure.jpg}
\caption{title}
\end{figure}

    \subsubsection{2.3 Case: Financial Time Series Analysis of S\&P500
Index}\label{case-financial-time-series-analysis-of-sp500-index}

\paragraph{Dataset:}\label{dataset}

The \textbf{S\&P500} index dataset can be obtained from yahoo finance
https://sg.finance.yahoo.com/quote/\%5EGSPC/history?p=\%5EGSPC.

\paragraph{References:}\label{references}

{[}1{]} Haroon, D. Python Machine Learning Case Studies. Apress..
{[}2{]} \texttt{statsmodels} module:
https://www.statsmodels.org/dev/tsa.html

    \begin{Verbatim}[commandchars=\\\{\}]
{\color{incolor}In [{\color{incolor}2}]:} \PY{c+c1}{\PYZsh{} Load dataset: You need to download s\PYZam{}p500 dataset from yahoo finance}
        \PY{o}{\PYZpc{}}\PY{k}{pwd}
        \PY{n}{sp500} \PY{o}{=} \PY{n}{pd}\PY{o}{.}\PY{n}{read\PYZus{}csv}\PY{p}{(}\PY{l+s+s1}{\PYZsq{}}\PY{l+s+s1}{./\PYZca{}GSPC.csv}\PY{l+s+s1}{\PYZsq{}}\PY{p}{)}\PY{p}{[}\PY{p}{[}\PY{l+s+s1}{\PYZsq{}}\PY{l+s+s1}{Date}\PY{l+s+s1}{\PYZsq{}}\PY{p}{,}\PY{l+s+s1}{\PYZsq{}}\PY{l+s+s1}{Adj Close}\PY{l+s+s1}{\PYZsq{}}\PY{p}{]}\PY{p}{]}
        
        \PY{c+c1}{\PYZsh{} Transform Date column to date type}
        \PY{n}{sp500}\PY{p}{[}\PY{l+s+s1}{\PYZsq{}}\PY{l+s+s1}{Date}\PY{l+s+s1}{\PYZsq{}}\PY{p}{]} \PY{o}{=} \PY{n}{sp500}\PY{p}{[}\PY{l+s+s1}{\PYZsq{}}\PY{l+s+s1}{Date}\PY{l+s+s1}{\PYZsq{}}\PY{p}{]}\PY{o}{.}\PY{n}{apply}\PY{p}{(}\PY{n+nb}{str}\PY{p}{)}
        \PY{n}{sp500}\PY{p}{[}\PY{l+s+s1}{\PYZsq{}}\PY{l+s+s1}{Date}\PY{l+s+s1}{\PYZsq{}}\PY{p}{]} \PY{o}{=} \PY{n}{pd}\PY{o}{.}\PY{n}{to\PYZus{}datetime}\PY{p}{(}\PY{n}{sp500}\PY{p}{[}\PY{l+s+s1}{\PYZsq{}}\PY{l+s+s1}{Date}\PY{l+s+s1}{\PYZsq{}}\PY{p}{]}\PY{p}{,} \PY{n}{infer\PYZus{}datetime\PYZus{}format}\PY{o}{=}\PY{k+kc}{True}\PY{p}{)}
        \PY{n}{sp500}\PY{o}{.}\PY{n}{head}\PY{p}{(}\PY{n}{n}\PY{o}{=}\PY{l+m+mi}{10}\PY{p}{)}
\end{Verbatim}


\begin{Verbatim}[commandchars=\\\{\}]
{\color{outcolor}Out[{\color{outcolor}2}]:}         Date    Adj Close
        0 2017-01-03  2257.830078
        1 2017-01-04  2270.750000
        2 2017-01-05  2269.000000
        3 2017-01-06  2276.979980
        4 2017-01-09  2268.899902
        5 2017-01-10  2268.899902
        6 2017-01-11  2275.320068
        7 2017-01-12  2270.439941
        8 2017-01-13  2274.639893
        9 2017-01-17  2267.889893
\end{Verbatim}
            
    \begin{Verbatim}[commandchars=\\\{\}]
{\color{incolor}In [{\color{incolor}3}]:} \PY{c+c1}{\PYZsh{} Split the dataset: First 70\PYZpc{} for training, Last 30\PYZpc{} for testing}
        \PY{n}{data\PYZus{}size} \PY{o}{=} \PY{n}{sp500}\PY{o}{.}\PY{n}{shape}\PY{p}{[}\PY{l+m+mi}{0}\PY{p}{]}
        
        \PY{n}{data\PYZus{}train} \PY{o}{=} \PY{n}{sp500}\PY{o}{.}\PY{n}{loc}\PY{p}{[}\PY{p}{:}\PY{n}{data\PYZus{}size}\PY{o}{*}\PY{l+m+mf}{0.7}\PY{p}{]}
        \PY{n}{data\PYZus{}test} \PY{o}{=} \PY{n}{sp500}\PY{o}{.}\PY{n}{loc}\PY{p}{[}\PY{n}{data\PYZus{}size}\PY{o}{*}\PY{l+m+mf}{0.7}\PY{p}{:}\PY{p}{]}
\end{Verbatim}


    \paragraph{2.3.1 Data Exploration}\label{data-exploration}

    \begin{Verbatim}[commandchars=\\\{\}]
{\color{incolor}In [{\color{incolor}4}]:} \PY{c+c1}{\PYZsh{} Plotting Adjusted Closing Index of Yahoo Stock on a Time Series}
        \PY{n}{plt}\PY{o}{.}\PY{n}{figure}\PY{p}{(}\PY{n}{figsize}\PY{o}{=}\PY{p}{(}\PY{l+m+mi}{15}\PY{p}{,}\PY{l+m+mi}{5}\PY{p}{)}\PY{p}{)}
        \PY{n}{plt}\PY{o}{.}\PY{n}{grid}\PY{p}{(}\PY{k+kc}{True}\PY{p}{)}
        \PY{n}{plt}\PY{o}{.}\PY{n}{plot}\PY{p}{(}\PY{n}{sp500}\PY{p}{[}\PY{l+s+s1}{\PYZsq{}}\PY{l+s+s1}{Date}\PY{l+s+s1}{\PYZsq{}}\PY{p}{]}\PY{p}{,} \PY{n}{sp500}\PY{p}{[}\PY{l+s+s1}{\PYZsq{}}\PY{l+s+s1}{Adj Close}\PY{l+s+s1}{\PYZsq{}}\PY{p}{]}\PY{p}{)}
        \PY{n}{plt}\PY{o}{.}\PY{n}{xlabel}\PY{p}{(}\PY{l+s+s1}{\PYZsq{}}\PY{l+s+s1}{Date}\PY{l+s+s1}{\PYZsq{}}\PY{p}{)}
        \PY{n}{plt}\PY{o}{.}\PY{n}{ylabel}\PY{p}{(}\PY{l+s+s1}{\PYZsq{}}\PY{l+s+s1}{Adjusted Closing Index}\PY{l+s+s1}{\PYZsq{}}\PY{p}{)}
        \PY{n}{plt}\PY{o}{.}\PY{n}{title}\PY{p}{(}\PY{l+s+s1}{\PYZsq{}}\PY{l+s+s1}{Yahoo Adjusted Closing Index \PYZhy{} Year 2017}\PY{l+s+s1}{\PYZsq{}}\PY{p}{)}
        \PY{n}{locs}\PY{p}{,} \PY{n}{labels} \PY{o}{=} \PY{n}{plt}\PY{o}{.}\PY{n}{xticks}\PY{p}{(}\PY{p}{)}
        \PY{n}{plt}\PY{o}{.}\PY{n}{setp}\PY{p}{(}\PY{n}{labels}\PY{p}{,} \PY{n}{rotation}\PY{o}{=}\PY{l+m+mi}{45}\PY{p}{)}
        \PY{n}{plt}\PY{o}{.}\PY{n}{show}\PY{p}{(}\PY{p}{)}
\end{Verbatim}


    \begin{center}
    \adjustimage{max size={0.9\linewidth}{0.9\paperheight}}{output_10_0.png}
    \end{center}
    { \hspace*{\fill} \\}
    
    \paragraph{Notice whether this time series data is stationary or not.
What about trend, seasonality,
autocorrelation?}\label{notice-whether-this-time-series-data-is-stationary-or-not.-what-about-trend-seasonality-autocorrelation}

    \paragraph{2.3.2 Test on Stationary
Condition}\label{test-on-stationary-condition}

\textbf{Dickey-Fuller test}: *
https://en.wikipedia.org/wiki/Dickey\%E2\%80\%93Fuller\_test *
https://en.wikipedia.org/wiki/Augmented\_Dickey\%E2\%80\%93Fuller\_test
*
http://www.ams.sunysb.edu/\textasciitilde{}zhu/ams586/UnitRoot\_ADF.pdf

    \begin{Verbatim}[commandchars=\\\{\}]
{\color{incolor}In [{\color{incolor}5}]:} \PY{n}{t}\PY{o}{=}\PY{l+m+mi}{15} \PY{c+c1}{\PYZsh{} Rolling time window, and calculate mean and standard error}
        \PY{n}{rolmean} \PY{o}{=} \PY{n}{data\PYZus{}train}\PY{p}{[}\PY{l+s+s1}{\PYZsq{}}\PY{l+s+s1}{Adj Close}\PY{l+s+s1}{\PYZsq{}}\PY{p}{]}\PY{o}{.}\PY{n}{rolling}\PY{p}{(}\PY{n}{window}\PY{o}{=}\PY{n}{t}\PY{p}{)}\PY{o}{.}\PY{n}{mean}\PY{p}{(}\PY{p}{)}
        \PY{n}{rolstd} \PY{o}{=} \PY{n}{data\PYZus{}train}\PY{p}{[}\PY{l+s+s1}{\PYZsq{}}\PY{l+s+s1}{Adj Close}\PY{l+s+s1}{\PYZsq{}}\PY{p}{]}\PY{o}{.}\PY{n}{rolling}\PY{p}{(}\PY{n}{window}\PY{o}{=}\PY{n}{t}\PY{p}{)}\PY{o}{.}\PY{n}{std}\PY{p}{(}\PY{p}{)}
        \PY{n}{rolmean}\PY{p}{,} \PY{n}{rolstd}
\end{Verbatim}


\begin{Verbatim}[commandchars=\\\{\}]
{\color{outcolor}Out[{\color{outcolor}5}]:} (0              NaN
         1              NaN
         2              NaN
         3              NaN
         4              NaN
         5              NaN
         6              NaN
         7              NaN
         8              NaN
         9              NaN
         10             NaN
         11             NaN
         12             NaN
         13             NaN
         14     2270.187305
         15     2272.889974
         16     2274.618636
         17     2276.331299
         18     2276.592627
         19     2277.257308
         20     2277.967318
         21     2278.335986
         22     2280.134652
         23     2281.329329
         24     2283.008675
         25     2284.527344
         26     2287.472689
         27     2290.458691
         28     2294.662028
         29     2298.496029
                   {\ldots}     
         146    2469.552018
         147    2471.174007
         148    2472.344678
         149    2473.796012
         150    2474.750000
         151    2474.762663
         152    2472.413330
         153    2470.331999
         154    2470.060677
         155    2469.226025
         156    2468.578027
         157    2465.550700
         158    2462.447363
         159    2459.652034
         160    2458.062695
         161    2455.827360
         162    2453.614697
         163    2451.362695
         164    2448.918034
         165    2447.010042
         166    2445.914714
         167    2448.144043
         168    2450.492708
         169    2449.960042
         170    2450.022038
         171    2449.821371
         172    2451.916032
         173    2456.086703
         174    2460.627360
         175    2463.684701
         Name: Adj Close, Length: 176, dtype: float64, 0            NaN
         1            NaN
         2            NaN
         3            NaN
         4            NaN
         5            NaN
         6            NaN
         7            NaN
         8            NaN
         9            NaN
         10           NaN
         11           NaN
         12           NaN
         13           NaN
         14      5.505278
         15      8.264886
         16     10.256967
         17     11.339455
         18     11.400477
         19     11.208970
         20     10.976673
         21     10.974277
         22     11.769840
         23     12.077716
         24     11.824176
         25     11.756815
         26     11.697618
         27     12.929022
         28     14.306325
         29     17.472291
                  {\ldots}    
         146     8.809176
         147     6.447136
         148     5.679833
         149     4.773801
         150     3.079797
         151     3.076133
         152     9.942918
         153    12.777934
         154    12.830637
         155    12.744855
         156    12.521284
         157    15.807244
         158    18.729012
         159    20.516977
         160    20.049121
         161    19.582587
         162    19.480154
         163    18.534244
         164    16.684689
         165    15.055737
         166    13.463855
         167    14.799236
         168    16.352882
         169    15.942234
         170    16.004959
         171    15.779253
         172    15.029120
         173    15.847872
         174    17.051005
         175    19.436213
         Name: Adj Close, Length: 176, dtype: float64)
\end{Verbatim}
            
    \begin{Verbatim}[commandchars=\\\{\}]
{\color{incolor}In [{\color{incolor}6}]:} \PY{c+c1}{\PYZsh{} Visualize whether mean and standard error of time series are stationary over time}
        \PY{n}{plt}\PY{o}{.}\PY{n}{figure}\PY{p}{(}\PY{n}{figsize}\PY{o}{=}\PY{p}{(}\PY{l+m+mi}{15}\PY{p}{,}\PY{l+m+mi}{5}\PY{p}{)}\PY{p}{)}
        \PY{n}{orig} \PY{o}{=} \PY{n}{plt}\PY{o}{.}\PY{n}{plot}\PY{p}{(}\PY{n}{data\PYZus{}train}\PY{p}{[}\PY{l+s+s1}{\PYZsq{}}\PY{l+s+s1}{Date}\PY{l+s+s1}{\PYZsq{}}\PY{p}{]}\PY{p}{,} \PY{n}{data\PYZus{}train}\PY{p}{[}\PY{l+s+s1}{\PYZsq{}}\PY{l+s+s1}{Adj Close}\PY{l+s+s1}{\PYZsq{}}\PY{p}{]}\PY{p}{,} \PY{n}{color}\PY{o}{=}\PY{l+s+s1}{\PYZsq{}}\PY{l+s+s1}{blue}\PY{l+s+s1}{\PYZsq{}}\PY{p}{,}\PY{n}{label}\PY{o}{=}\PY{l+s+s1}{\PYZsq{}}\PY{l+s+s1}{Original}\PY{l+s+s1}{\PYZsq{}}\PY{p}{)}
        \PY{n}{mean} \PY{o}{=} \PY{n}{plt}\PY{o}{.}\PY{n}{plot}\PY{p}{(}\PY{n}{data\PYZus{}train}\PY{p}{[}\PY{l+s+s1}{\PYZsq{}}\PY{l+s+s1}{Date}\PY{l+s+s1}{\PYZsq{}}\PY{p}{]}\PY{p}{,} \PY{n}{rolmean}\PY{p}{,} \PY{n}{color}\PY{o}{=}\PY{l+s+s1}{\PYZsq{}}\PY{l+s+s1}{red}\PY{l+s+s1}{\PYZsq{}}\PY{p}{,} \PY{n}{label}\PY{o}{=}\PY{l+s+s1}{\PYZsq{}}\PY{l+s+s1}{Rolling Mean}\PY{l+s+s1}{\PYZsq{}}\PY{p}{)}
        \PY{n}{std} \PY{o}{=} \PY{n}{plt}\PY{o}{.}\PY{n}{plot}\PY{p}{(}\PY{n}{data\PYZus{}train}\PY{p}{[}\PY{l+s+s1}{\PYZsq{}}\PY{l+s+s1}{Date}\PY{l+s+s1}{\PYZsq{}}\PY{p}{]}\PY{p}{,} \PY{n}{rolstd}\PY{p}{,} \PY{n}{color}\PY{o}{=}\PY{l+s+s1}{\PYZsq{}}\PY{l+s+s1}{black}\PY{l+s+s1}{\PYZsq{}}\PY{p}{,} \PY{n}{label} \PY{o}{=} \PY{l+s+s1}{\PYZsq{}}\PY{l+s+s1}{Rolling Std}\PY{l+s+s1}{\PYZsq{}}\PY{p}{)}
        \PY{n}{plt}\PY{o}{.}\PY{n}{legend}\PY{p}{(}\PY{n}{loc}\PY{o}{=}\PY{l+s+s1}{\PYZsq{}}\PY{l+s+s1}{best}\PY{l+s+s1}{\PYZsq{}}\PY{p}{)}
        \PY{n}{plt}\PY{o}{.}\PY{n}{title}\PY{p}{(}\PY{l+s+s1}{\PYZsq{}}\PY{l+s+s1}{Rolling Mean \PYZam{} Standard Deviation}\PY{l+s+s1}{\PYZsq{}}\PY{p}{)}
        \PY{n}{plt}\PY{o}{.}\PY{n}{show}\PY{p}{(}\PY{n}{block}\PY{o}{=}\PY{k+kc}{False}\PY{p}{)}
\end{Verbatim}


    \begin{center}
    \adjustimage{max size={0.9\linewidth}{0.9\paperheight}}{output_14_0.png}
    \end{center}
    { \hspace*{\fill} \\}
    
    \paragraph{It seems that standard error remains constant, but mean is
changing.}\label{it-seems-that-standard-error-remains-constant-but-mean-is-changing.}

    \begin{Verbatim}[commandchars=\\\{\}]
{\color{incolor}In [{\color{incolor}7}]:} \PY{c+c1}{\PYZsh{} Dickey\PYZhy{}Fuller test on stationary condition}
        \PY{n+nb}{print} \PY{p}{(}\PY{l+s+s1}{\PYZsq{}}\PY{l+s+s1}{Results of Dickey\PYZhy{}Fuller Test:}\PY{l+s+s1}{\PYZsq{}}\PY{p}{)}
        
        \PY{c+c1}{\PYZsh{} Dickey\PYZhy{}Fuller test}
        \PY{n}{dftest} \PY{o}{=} \PY{n}{adfuller}\PY{p}{(}\PY{n}{data\PYZus{}train}\PY{p}{[}\PY{l+s+s1}{\PYZsq{}}\PY{l+s+s1}{Adj Close}\PY{l+s+s1}{\PYZsq{}}\PY{p}{]}\PY{p}{,} \PY{n}{autolag}\PY{o}{=}\PY{k+kc}{None}\PY{p}{)} 
        \PY{n}{dfoutput} \PY{o}{=} \PY{n}{pd}\PY{o}{.}\PY{n}{Series}\PY{p}{(}\PY{n}{dftest}\PY{p}{[}\PY{l+m+mi}{0}\PY{p}{:}\PY{l+m+mi}{4}\PY{p}{]}\PY{p}{,} \PY{n}{index}\PY{o}{=}\PY{p}{[}\PY{l+s+s1}{\PYZsq{}}\PY{l+s+s1}{Test Statistic}\PY{l+s+s1}{\PYZsq{}}\PY{p}{,}\PY{l+s+s1}{\PYZsq{}}\PY{l+s+s1}{p\PYZhy{}value}\PY{l+s+s1}{\PYZsq{}}\PY{p}{,}\PY{l+s+s1}{\PYZsq{}}\PY{l+s+s1}{\PYZsh{}Lags Used}\PY{l+s+s1}{\PYZsq{}}\PY{p}{,}\PY{l+s+s1}{\PYZsq{}}\PY{l+s+s1}{Number of Observations Used}\PY{l+s+s1}{\PYZsq{}}\PY{p}{]}\PY{p}{)}
        
        \PY{k}{for} \PY{n}{key}\PY{p}{,} \PY{n}{value} \PY{o+ow}{in} \PY{n}{dftest}\PY{p}{[}\PY{l+m+mi}{4}\PY{p}{]}\PY{o}{.}\PY{n}{items}\PY{p}{(}\PY{p}{)}\PY{p}{:}
            \PY{n}{dfoutput}\PY{p}{[}\PY{l+s+s1}{\PYZsq{}}\PY{l+s+s1}{Critical Value (}\PY{l+s+si}{\PYZpc{}s}\PY{l+s+s1}{)}\PY{l+s+s1}{\PYZsq{}}\PY{o}{\PYZpc{}}\PY{k}{key}] = value
        
        \PY{n+nb}{print} \PY{p}{(}\PY{n}{dfoutput}\PY{p}{)}
\end{Verbatim}


    \begin{Verbatim}[commandchars=\\\{\}]
Results of Dickey-Fuller Test:
Test Statistic                  -1.442525
p-value                          0.561728
\#Lags Used                      14.000000
Number of Observations Used    161.000000
Critical Value (1\%)             -3.471633
Critical Value (5\%)             -2.879665
Critical Value (10\%)            -2.576434
dtype: float64

    \end{Verbatim}

    \paragraph{Dickey-Fuller test shows that null hypothesis
(nonstationarity) cannot be rejected, so the time series may be
nonstationary. If non-stationary, we cannot use AR(I)MA to fit the data,
make inference or do forecasting. Thus, we need to make the time series
stationary
first.}\label{dickey-fuller-test-shows-that-null-hypothesis-nonstationarity-cannot-be-rejected-so-the-time-series-may-be-nonstationary.-if-non-stationary-we-cannot-use-arima-to-fit-the-data-make-inference-or-do-forecasting.-thus-we-need-to-make-the-time-series-stationary-first.}

\paragraph{We have several ways to make time series
stationary.}\label{we-have-several-ways-to-make-time-series-stationary.}

    \paragraph{2.3.3 Make Time Series Stationary: Eliminating Trend and
Seasonality}\label{make-time-series-stationary-eliminating-trend-and-seasonality}

\textbf{Method 1. Estimating Trend and Removing It from the Original
Series: Using Moving Average Smoothing}

Moving average smoothing is similar to rolling mean in that in both
these methods, we calculate the average of observations lying within a
fixed window size, in order to estimate the trending at a given time.

    \begin{Verbatim}[commandchars=\\\{\}]
{\color{incolor}In [{\color{incolor}8}]:} \PY{c+c1}{\PYZsh{} Applying Moving Average Smoothing to the Time Series Object to Plot the Trend}
        \PY{n}{t} \PY{o}{=} \PY{l+m+mi}{15} \PY{c+c1}{\PYZsh{} Time window for Moving Average Smoothing Trend}
        \PY{n}{moving\PYZus{}avg} \PY{o}{=} \PY{n}{data\PYZus{}train}\PY{p}{[}\PY{l+s+s1}{\PYZsq{}}\PY{l+s+s1}{Adj Close}\PY{l+s+s1}{\PYZsq{}}\PY{p}{]}\PY{o}{.}\PY{n}{rolling}\PY{p}{(}\PY{n}{window}\PY{o}{=}\PY{n}{t}\PY{p}{)}\PY{o}{.}\PY{n}{mean}\PY{p}{(}\PY{p}{)} \PY{c+c1}{\PYZsh{} Trend}
        \PY{n}{plt}\PY{o}{.}\PY{n}{figure}\PY{p}{(}\PY{n}{figsize}\PY{o}{=}\PY{p}{(}\PY{l+m+mi}{15}\PY{p}{,}\PY{l+m+mi}{5}\PY{p}{)}\PY{p}{)}
        \PY{n}{plt}\PY{o}{.}\PY{n}{plot}\PY{p}{(}\PY{n}{data\PYZus{}train}\PY{p}{[}\PY{l+s+s1}{\PYZsq{}}\PY{l+s+s1}{Date}\PY{l+s+s1}{\PYZsq{}}\PY{p}{]}\PY{p}{,} \PY{n}{data\PYZus{}train}\PY{p}{[}\PY{l+s+s1}{\PYZsq{}}\PY{l+s+s1}{Adj Close}\PY{l+s+s1}{\PYZsq{}}\PY{p}{]}\PY{p}{,} \PY{n}{label}\PY{o}{=}\PY{l+s+s1}{\PYZsq{}}\PY{l+s+s1}{Actual S\PYZam{}P500 Index}\PY{l+s+s1}{\PYZsq{}}\PY{p}{)}
        \PY{n}{plt}\PY{o}{.}\PY{n}{plot}\PY{p}{(}\PY{n}{data\PYZus{}train}\PY{p}{[}\PY{l+s+s1}{\PYZsq{}}\PY{l+s+s1}{Date}\PY{l+s+s1}{\PYZsq{}}\PY{p}{]}\PY{p}{,} \PY{n}{moving\PYZus{}avg}\PY{p}{,} \PY{n}{color}\PY{o}{=}\PY{l+s+s1}{\PYZsq{}}\PY{l+s+s1}{red}\PY{l+s+s1}{\PYZsq{}}\PY{p}{,} \PY{n}{label}\PY{o}{=}\PY{l+s+s1}{\PYZsq{}}\PY{l+s+s1}{Trend: Rolling Mean}\PY{l+s+s1}{\PYZsq{}}\PY{p}{)}
        \PY{n}{plt}\PY{o}{.}\PY{n}{title}\PY{p}{(}\PY{l+s+s1}{\PYZsq{}}\PY{l+s+s1}{Estimated Trend: Using Moving Average Smoothing}\PY{l+s+s1}{\PYZsq{}}\PY{p}{)}
        \PY{n}{plt}\PY{o}{.}\PY{n}{legend}\PY{p}{(}\PY{n}{loc}\PY{o}{=}\PY{l+s+s1}{\PYZsq{}}\PY{l+s+s1}{best}\PY{l+s+s1}{\PYZsq{}}\PY{p}{)}
        \PY{n}{plt}\PY{o}{.}\PY{n}{show}\PY{p}{(}\PY{p}{)}
\end{Verbatim}


    \begin{center}
    \adjustimage{max size={0.9\linewidth}{0.9\paperheight}}{output_19_0.png}
    \end{center}
    { \hspace*{\fill} \\}
    
    \begin{Verbatim}[commandchars=\\\{\}]
{\color{incolor}In [{\color{incolor}9}]:} \PY{c+c1}{\PYZsh{} Remove trend from original time series: Subtract this estimated trend}
        \PY{n}{data\PYZus{}log\PYZus{}moving\PYZus{}avg\PYZus{}diff} \PY{o}{=} \PY{n}{data\PYZus{}train}\PY{p}{[}\PY{l+s+s1}{\PYZsq{}}\PY{l+s+s1}{Adj Close}\PY{l+s+s1}{\PYZsq{}}\PY{p}{]} \PY{o}{\PYZhy{}} \PY{n}{moving\PYZus{}avg}
        \PY{n}{data\PYZus{}log\PYZus{}moving\PYZus{}avg\PYZus{}diff}\PY{o}{.}\PY{n}{head}\PY{p}{(}\PY{n}{n}\PY{o}{=}\PY{l+m+mi}{20}\PY{p}{)}
\end{Verbatim}


\begin{Verbatim}[commandchars=\\\{\}]
{\color{outcolor}Out[{\color{outcolor}9}]:} 0           NaN
        1           NaN
        2           NaN
        3           NaN
        4           NaN
        5           NaN
        6           NaN
        7           NaN
        8           NaN
        9           NaN
        10          NaN
        11          NaN
        12          NaN
        13          NaN
        14     9.882763
        15    25.480143
        16    22.061296
        17    18.358642
        18     4.307275
        19     1.612809
        Name: Adj Close, dtype: float64
\end{Verbatim}
            
    \begin{Verbatim}[commandchars=\\\{\}]
{\color{incolor}In [{\color{incolor}10}]:} \PY{c+c1}{\PYZsh{} Drop missing values}
         \PY{n}{data\PYZus{}log\PYZus{}moving\PYZus{}avg\PYZus{}diff}\PY{o}{.}\PY{n}{dropna}\PY{p}{(}\PY{n}{inplace}\PY{o}{=}\PY{k+kc}{True}\PY{p}{)}
         \PY{n}{data\PYZus{}log\PYZus{}moving\PYZus{}avg\PYZus{}diff}\PY{o}{.}\PY{n}{head}\PY{p}{(}\PY{n}{n}\PY{o}{=}\PY{l+m+mi}{20}\PY{p}{)}
\end{Verbatim}


\begin{Verbatim}[commandchars=\\\{\}]
{\color{outcolor}Out[{\color{outcolor}10}]:} 14     9.882763
         15    25.480143
         16    22.061296
         17    18.358642
         18     4.307275
         19     1.612809
         20     1.582731
         21     2.514112
         22    17.285270
         23    11.230730
         24    10.071403
         25    10.142578
         26    20.397428
         27    25.641407
         28    33.587972
         29    39.084049
         30    47.361979
         31    41.962614
         32    42.137890
         33    50.725863
         Name: Adj Close, dtype: float64
\end{Verbatim}
            
    \begin{Verbatim}[commandchars=\\\{\}]
{\color{incolor}In [{\color{incolor}11}]:} \PY{c+c1}{\PYZsh{} Evaluating Trendless Time Series for Stationary: Using Dickey\PYZhy{}Fuller test}
         \PY{n+nb}{print} \PY{p}{(}\PY{l+s+s1}{\PYZsq{}}\PY{l+s+s1}{Results of Dickey\PYZhy{}Fuller Test:}\PY{l+s+s1}{\PYZsq{}}\PY{p}{)}
         
         \PY{c+c1}{\PYZsh{} Dickey\PYZhy{}Fuller test}
         \PY{n}{dftest} \PY{o}{=} \PY{n}{adfuller}\PY{p}{(}\PY{n}{data\PYZus{}log\PYZus{}moving\PYZus{}avg\PYZus{}diff}\PY{p}{,} \PY{n}{autolag}\PY{o}{=}\PY{k+kc}{None}\PY{p}{)} 
         \PY{n}{dfoutput} \PY{o}{=} \PY{n}{pd}\PY{o}{.}\PY{n}{Series}\PY{p}{(}\PY{n}{dftest}\PY{p}{[}\PY{l+m+mi}{0}\PY{p}{:}\PY{l+m+mi}{4}\PY{p}{]}\PY{p}{,} \PY{n}{index}\PY{o}{=}\PY{p}{[}\PY{l+s+s1}{\PYZsq{}}\PY{l+s+s1}{Test Statistic}\PY{l+s+s1}{\PYZsq{}}\PY{p}{,}\PY{l+s+s1}{\PYZsq{}}\PY{l+s+s1}{p\PYZhy{}value}\PY{l+s+s1}{\PYZsq{}}\PY{p}{,}\PY{l+s+s1}{\PYZsq{}}\PY{l+s+s1}{\PYZsh{}Lags Used}\PY{l+s+s1}{\PYZsq{}}\PY{p}{,}\PY{l+s+s1}{\PYZsq{}}\PY{l+s+s1}{Number of Observations Used}\PY{l+s+s1}{\PYZsq{}}\PY{p}{]}\PY{p}{)}
         
         \PY{k}{for} \PY{n}{key}\PY{p}{,} \PY{n}{value} \PY{o+ow}{in} \PY{n}{dftest}\PY{p}{[}\PY{l+m+mi}{4}\PY{p}{]}\PY{o}{.}\PY{n}{items}\PY{p}{(}\PY{p}{)}\PY{p}{:}
             \PY{n}{dfoutput}\PY{p}{[}\PY{l+s+s1}{\PYZsq{}}\PY{l+s+s1}{Critical Value (}\PY{l+s+si}{\PYZpc{}s}\PY{l+s+s1}{)}\PY{l+s+s1}{\PYZsq{}}\PY{o}{\PYZpc{}}\PY{k}{key}] = value
         
         \PY{n+nb}{print} \PY{p}{(}\PY{n}{dfoutput}\PY{p}{)}
\end{Verbatim}


    \begin{Verbatim}[commandchars=\\\{\}]
Results of Dickey-Fuller Test:
Test Statistic                  -3.566682
p-value                          0.006431
\#Lags Used                      14.000000
Number of Observations Used    147.000000
Critical Value (1\%)             -3.475637
Critical Value (5\%)             -2.881410
Critical Value (10\%)            -2.577365
dtype: float64

    \end{Verbatim}

    \paragraph{This time, we find that Dickey-Fuller test is significant
(p-value\textless{}5\%). This means, after removing general trend, the
time series data probably become
stationary.}\label{this-time-we-find-that-dickey-fuller-test-is-significant-p-value5.-this-means-after-removing-general-trend-the-time-series-data-probably-become-stationary.}

    \textbf{Method 2. Differencing}

Time series datasets may contain both \textbf{trends} and
\textbf{seasonality}, which may need to be removed prior to modeling,
otherwise we cannot get stationary data. Trends can result in a varying
mean over time, whereas seasonality can result in a changing variance
over time, both which define a time series as being non-stationary.
Stationary datasets are those that have a stable mean and variance, and
are in turn much easier to model.

Differencing is a popular and widely used data transformation for making
time series data stationary. Differencing subtracts the time series from
a lagged version of itself (i.e., observations at the previous
instances). Differencing stabilizes the mean of a time series by
removing changes in the level. Differencing, given its definition, is
also an effective way to make the time series object stationary.

We begin with first-order differencing; that is, a lag of 1 was induced
while differencing. We can also try if a second- or third-order
differencing can further improve the stationary nature of the time
series object.

    \begin{Verbatim}[commandchars=\\\{\}]
{\color{incolor}In [{\color{incolor}12}]:} \PY{c+c1}{\PYZsh{} Applying First\PYZhy{}Order Differencing to the original time series data}
         \PY{n}{data\PYZus{}diff} \PY{o}{=} \PY{n}{data\PYZus{}train}\PY{p}{[}\PY{l+s+s1}{\PYZsq{}}\PY{l+s+s1}{Adj Close}\PY{l+s+s1}{\PYZsq{}}\PY{p}{]} \PY{o}{\PYZhy{}} \PY{n}{data\PYZus{}train}\PY{p}{[}\PY{l+s+s1}{\PYZsq{}}\PY{l+s+s1}{Adj Close}\PY{l+s+s1}{\PYZsq{}}\PY{p}{]}\PY{o}{.}\PY{n}{shift}\PY{p}{(}\PY{l+m+mi}{1}\PY{p}{)}
         \PY{n}{data\PYZus{}diff}\PY{o}{.}\PY{n}{dropna}\PY{p}{(}\PY{n}{inplace}\PY{o}{=}\PY{k+kc}{True}\PY{p}{)}
         
         \PY{n}{t}\PY{o}{=}\PY{l+m+mi}{30} \PY{c+c1}{\PYZsh{} Rolling time window, and calculate mean and standard error}
         \PY{n}{rolmean} \PY{o}{=} \PY{n}{data\PYZus{}diff}\PY{o}{.}\PY{n}{rolling}\PY{p}{(}\PY{n}{window}\PY{o}{=}\PY{n}{t}\PY{p}{)}\PY{o}{.}\PY{n}{mean}\PY{p}{(}\PY{p}{)}
         \PY{n}{rolstd} \PY{o}{=} \PY{n}{data\PYZus{}diff}\PY{o}{.}\PY{n}{rolling}\PY{p}{(}\PY{n}{window}\PY{o}{=}\PY{n}{t}\PY{p}{)}\PY{o}{.}\PY{n}{std}\PY{p}{(}\PY{p}{)}
         
         \PY{c+c1}{\PYZsh{} Visualize whether mean and standard error of time series are stationary over time}
         \PY{n}{plt}\PY{o}{.}\PY{n}{figure}\PY{p}{(}\PY{n}{figsize}\PY{o}{=}\PY{p}{(}\PY{l+m+mi}{15}\PY{p}{,}\PY{l+m+mi}{5}\PY{p}{)}\PY{p}{)}
         \PY{c+c1}{\PYZsh{}orig = plt.plot(data\PYZus{}train[\PYZsq{}Adj Close\PYZsq{}], color=\PYZsq{}blue\PYZsq{},label=\PYZsq{}Original\PYZsq{})}
         \PY{n}{diff} \PY{o}{=} \PY{n}{plt}\PY{o}{.}\PY{n}{plot}\PY{p}{(}\PY{n}{data\PYZus{}diff}\PY{p}{,} \PY{n}{color}\PY{o}{=}\PY{l+s+s1}{\PYZsq{}}\PY{l+s+s1}{green}\PY{l+s+s1}{\PYZsq{}}\PY{p}{,}\PY{n}{label}\PY{o}{=}\PY{l+s+s1}{\PYZsq{}}\PY{l+s+s1}{Differencing}\PY{l+s+s1}{\PYZsq{}}\PY{p}{)}
         \PY{n}{mean} \PY{o}{=} \PY{n}{plt}\PY{o}{.}\PY{n}{plot}\PY{p}{(}\PY{n}{rolmean}\PY{p}{,} \PY{n}{color}\PY{o}{=}\PY{l+s+s1}{\PYZsq{}}\PY{l+s+s1}{red}\PY{l+s+s1}{\PYZsq{}}\PY{p}{,} \PY{n}{label}\PY{o}{=}\PY{l+s+s1}{\PYZsq{}}\PY{l+s+s1}{Rolling Mean}\PY{l+s+s1}{\PYZsq{}}\PY{p}{)}
         \PY{n}{std} \PY{o}{=} \PY{n}{plt}\PY{o}{.}\PY{n}{plot}\PY{p}{(}\PY{n}{rolstd}\PY{p}{,} \PY{n}{color}\PY{o}{=}\PY{l+s+s1}{\PYZsq{}}\PY{l+s+s1}{black}\PY{l+s+s1}{\PYZsq{}}\PY{p}{,} \PY{n}{label} \PY{o}{=} \PY{l+s+s1}{\PYZsq{}}\PY{l+s+s1}{Rolling Std}\PY{l+s+s1}{\PYZsq{}}\PY{p}{)}
         \PY{n}{plt}\PY{o}{.}\PY{n}{legend}\PY{p}{(}\PY{n}{loc}\PY{o}{=}\PY{l+s+s1}{\PYZsq{}}\PY{l+s+s1}{best}\PY{l+s+s1}{\PYZsq{}}\PY{p}{)}
         \PY{n}{plt}\PY{o}{.}\PY{n}{title}\PY{p}{(}\PY{l+s+s1}{\PYZsq{}}\PY{l+s+s1}{Rolling Mean \PYZam{} Standard Deviation after Differencing}\PY{l+s+s1}{\PYZsq{}}\PY{p}{)}
         \PY{n}{plt}\PY{o}{.}\PY{n}{show}\PY{p}{(}\PY{n}{block}\PY{o}{=}\PY{k+kc}{False}\PY{p}{)}
         
         \PY{c+c1}{\PYZsh{} Dicky\PYZhy{}Fuller test on stationary condition}
         \PY{n+nb}{print} \PY{p}{(}\PY{l+s+s1}{\PYZsq{}}\PY{l+s+s1}{Results of Dickey\PYZhy{}Fuller Test:}\PY{l+s+s1}{\PYZsq{}}\PY{p}{)}
         
         \PY{c+c1}{\PYZsh{} Dickey\PYZhy{}Fuller test}
         \PY{n}{dftest} \PY{o}{=} \PY{n}{adfuller}\PY{p}{(}\PY{n}{data\PYZus{}diff}\PY{p}{,} \PY{n}{autolag}\PY{o}{=}\PY{k+kc}{None}\PY{p}{)} 
         \PY{n}{dfoutput} \PY{o}{=} \PY{n}{pd}\PY{o}{.}\PY{n}{Series}\PY{p}{(}\PY{n}{dftest}\PY{p}{[}\PY{l+m+mi}{0}\PY{p}{:}\PY{l+m+mi}{4}\PY{p}{]}\PY{p}{,} \PY{n}{index}\PY{o}{=}\PY{p}{[}\PY{l+s+s1}{\PYZsq{}}\PY{l+s+s1}{Test Statistic}\PY{l+s+s1}{\PYZsq{}}\PY{p}{,}\PY{l+s+s1}{\PYZsq{}}\PY{l+s+s1}{p\PYZhy{}value}\PY{l+s+s1}{\PYZsq{}}\PY{p}{,}\PY{l+s+s1}{\PYZsq{}}\PY{l+s+s1}{\PYZsh{}Lags Used}\PY{l+s+s1}{\PYZsq{}}\PY{p}{,}\PY{l+s+s1}{\PYZsq{}}\PY{l+s+s1}{Number of Observations Used}\PY{l+s+s1}{\PYZsq{}}\PY{p}{]}\PY{p}{)}
         
         \PY{k}{for} \PY{n}{key}\PY{p}{,} \PY{n}{value} \PY{o+ow}{in} \PY{n}{dftest}\PY{p}{[}\PY{l+m+mi}{4}\PY{p}{]}\PY{o}{.}\PY{n}{items}\PY{p}{(}\PY{p}{)}\PY{p}{:}
             \PY{n}{dfoutput}\PY{p}{[}\PY{l+s+s1}{\PYZsq{}}\PY{l+s+s1}{Critical Value (}\PY{l+s+si}{\PYZpc{}s}\PY{l+s+s1}{)}\PY{l+s+s1}{\PYZsq{}}\PY{o}{\PYZpc{}}\PY{k}{key}] = value
         
         \PY{n+nb}{print} \PY{p}{(}\PY{n}{dfoutput}\PY{p}{)}
\end{Verbatim}


    \begin{center}
    \adjustimage{max size={0.9\linewidth}{0.9\paperheight}}{output_25_0.png}
    \end{center}
    { \hspace*{\fill} \\}
    
    \begin{Verbatim}[commandchars=\\\{\}]
Results of Dickey-Fuller Test:
Test Statistic                  -3.266632
p-value                          0.016438
\#Lags Used                      14.000000
Number of Observations Used    160.000000
Critical Value (1\%)             -3.471896
Critical Value (5\%)             -2.879780
Critical Value (10\%)            -2.576495
dtype: float64

    \end{Verbatim}

    \paragraph{We see a remarkable improvement as the rolling mean and
rolling standard deviation after differencing appear to be almost
constant. The same was reflected by the Dickey-Fuller test as the test
statistics come out to be much less than the 5\% critical values. The
de-seasonality and detrending can make time series seem to be almost
stationary in nature
now.}\label{we-see-a-remarkable-improvement-as-the-rolling-mean-and-rolling-standard-deviation-after-differencing-appear-to-be-almost-constant.-the-same-was-reflected-by-the-dickey-fuller-test-as-the-test-statistics-come-out-to-be-much-less-than-the-5-critical-values.-the-de-seasonality-and-detrending-can-make-time-series-seem-to-be-almost-stationary-in-nature-now.}

    \textbf{Method 3. Decomposition}

We can also directly break the time series object into trend,
seasonality and residuals components by visualization. This leads to the
concept of decomposition. We hope that the residual components can be
closer to stationary after removing trend and seasonality.

Decomposition is an approach to dissect the time series data and
visualize the detailed components. This approach helps you to know how
to transform non-stationary time series into stationary ones. It does
that by dividing the time series into three components: \textbf{trend},
\textbf{seasonality}, and \textbf{residuals}. The component of interest
in this case is the residuals (i.e., time series without trend and
seasonality).

\texttt{statsmodels} module:
https://www.statsmodels.org/dev/generated/statsmodels.tsa.seasonal.seasonal\_decompose.html

    \begin{Verbatim}[commandchars=\\\{\}]
{\color{incolor}In [{\color{incolor}13}]:} \PY{c+c1}{\PYZsh{} Decomposing the Original Time Series}
         \PY{k+kn}{from} \PY{n+nn}{statsmodels}\PY{n+nn}{.}\PY{n+nn}{tsa}\PY{n+nn}{.}\PY{n+nn}{seasonal} \PY{k}{import} \PY{n}{seasonal\PYZus{}decompose}
         \PY{n}{decomposition} \PY{o}{=} \PY{n}{seasonal\PYZus{}decompose}\PY{p}{(}\PY{n+nb}{list}\PY{p}{(}\PY{n}{data\PYZus{}train}\PY{p}{[}\PY{l+s+s1}{\PYZsq{}}\PY{l+s+s1}{Adj Close}\PY{l+s+s1}{\PYZsq{}}\PY{p}{]}\PY{p}{)}\PY{p}{,} \PY{n}{freq}\PY{o}{=}\PY{l+m+mi}{15}\PY{p}{)}
         \PY{n}{trend} \PY{o}{=} \PY{n}{decomposition}\PY{o}{.}\PY{n}{trend}
         \PY{n}{seasonal} \PY{o}{=} \PY{n}{decomposition}\PY{o}{.}\PY{n}{seasonal}
         \PY{n}{residual} \PY{o}{=} \PY{n}{decomposition}\PY{o}{.}\PY{n}{resid}
         
         \PY{c+c1}{\PYZsh{} Plot the components}
         \PY{n}{plt}\PY{o}{.}\PY{n}{figure}\PY{p}{(}\PY{n}{figsize}\PY{o}{=}\PY{p}{(}\PY{l+m+mi}{15}\PY{p}{,}\PY{l+m+mi}{10}\PY{p}{)}\PY{p}{)}
         \PY{n}{plt}\PY{o}{.}\PY{n}{subplot}\PY{p}{(}\PY{l+m+mi}{411}\PY{p}{)}
         \PY{n}{plt}\PY{o}{.}\PY{n}{plot}\PY{p}{(}\PY{n}{data\PYZus{}train}\PY{p}{[}\PY{l+s+s1}{\PYZsq{}}\PY{l+s+s1}{Adj Close}\PY{l+s+s1}{\PYZsq{}}\PY{p}{]}\PY{p}{,} \PY{n}{label}\PY{o}{=}\PY{l+s+s1}{\PYZsq{}}\PY{l+s+s1}{Original}\PY{l+s+s1}{\PYZsq{}}\PY{p}{)}
         \PY{n}{plt}\PY{o}{.}\PY{n}{legend}\PY{p}{(}\PY{n}{loc}\PY{o}{=}\PY{l+s+s1}{\PYZsq{}}\PY{l+s+s1}{best}\PY{l+s+s1}{\PYZsq{}}\PY{p}{)}
         \PY{n}{plt}\PY{o}{.}\PY{n}{subplot}\PY{p}{(}\PY{l+m+mi}{412}\PY{p}{)}
         \PY{n}{plt}\PY{o}{.}\PY{n}{plot}\PY{p}{(}\PY{n}{trend}\PY{p}{,} \PY{n}{label}\PY{o}{=}\PY{l+s+s1}{\PYZsq{}}\PY{l+s+s1}{Trend}\PY{l+s+s1}{\PYZsq{}}\PY{p}{)}
         \PY{n}{plt}\PY{o}{.}\PY{n}{legend}\PY{p}{(}\PY{n}{loc}\PY{o}{=}\PY{l+s+s1}{\PYZsq{}}\PY{l+s+s1}{best}\PY{l+s+s1}{\PYZsq{}}\PY{p}{)}
         \PY{n}{plt}\PY{o}{.}\PY{n}{subplot}\PY{p}{(}\PY{l+m+mi}{413}\PY{p}{)}
         \PY{n}{plt}\PY{o}{.}\PY{n}{plot}\PY{p}{(}\PY{n}{seasonal}\PY{p}{,}\PY{n}{label}\PY{o}{=}\PY{l+s+s1}{\PYZsq{}}\PY{l+s+s1}{Seasonality}\PY{l+s+s1}{\PYZsq{}}\PY{p}{)}
         \PY{n}{plt}\PY{o}{.}\PY{n}{legend}\PY{p}{(}\PY{n}{loc}\PY{o}{=}\PY{l+s+s1}{\PYZsq{}}\PY{l+s+s1}{best}\PY{l+s+s1}{\PYZsq{}}\PY{p}{)}
         \PY{n}{plt}\PY{o}{.}\PY{n}{subplot}\PY{p}{(}\PY{l+m+mi}{414}\PY{p}{)}
         \PY{n}{plt}\PY{o}{.}\PY{n}{plot}\PY{p}{(}\PY{n}{residual}\PY{p}{,} \PY{n}{label}\PY{o}{=}\PY{l+s+s1}{\PYZsq{}}\PY{l+s+s1}{Residuals}\PY{l+s+s1}{\PYZsq{}}\PY{p}{)}
         \PY{n}{plt}\PY{o}{.}\PY{n}{legend}\PY{p}{(}\PY{n}{loc}\PY{o}{=}\PY{l+s+s1}{\PYZsq{}}\PY{l+s+s1}{best}\PY{l+s+s1}{\PYZsq{}}\PY{p}{)}
         \PY{n}{plt}\PY{o}{.}\PY{n}{tight\PYZus{}layout}\PY{p}{(}\PY{p}{)}
\end{Verbatim}


    \begin{center}
    \adjustimage{max size={0.9\linewidth}{0.9\paperheight}}{output_28_0.png}
    \end{center}
    { \hspace*{\fill} \\}
    
    \paragraph{We find that (1) trends are weakly linearly increasing in
nature, (2) seasonality may exist. The seasonality is a pattern that
repeats itself after a fixed time interval. We check the residual
component and find it to be fairly constant over time. Now we plan to
evaluate the residuals to see if they are stationary in nature or
not.}\label{we-find-that-1-trends-are-weakly-linearly-increasing-in-nature-2-seasonality-may-exist.-the-seasonality-is-a-pattern-that-repeats-itself-after-a-fixed-time-interval.-we-check-the-residual-component-and-find-it-to-be-fairly-constant-over-time.-now-we-plan-to-evaluate-the-residuals-to-see-if-they-are-stationary-in-nature-or-not.}

\paragraph{Note that after de-trending the time series, and you run
Dickey-Fuller test, you may find that the transformed time series have
already been stationary. This measn seasonality may not exert
significant effect in this case. So you can just fit the AR(I)MA model
on the stationary time series data. If it is still non-stationary, the
possible reason is that seasonality has significant effect (e.g.,
seasonal random walk). Then you need to remove seasonality from the time
series
data.}\label{note-that-after-de-trending-the-time-series-and-you-run-dickey-fuller-test-you-may-find-that-the-transformed-time-series-have-already-been-stationary.-this-measn-seasonality-may-not-exert-significant-effect-in-this-case.-so-you-can-just-fit-the-arima-model-on-the-stationary-time-series-data.-if-it-is-still-non-stationary-the-possible-reason-is-that-seasonality-has-significant-effect-e.g.-seasonal-random-walk.-then-you-need-to-remove-seasonality-from-the-time-series-data.}

\paragraph{Luckily, we have SARIMA (seasonal ARIMA) model that can
automatically fit the seasonality of time series data. If you are
interested in it, you can refer to the
link.}\label{luckily-we-have-sarima-seasonal-arima-model-that-can-automatically-fit-the-seasonality-of-time-series-data.-if-you-are-interested-in-it-you-can-refer-to-the-link.}

References:
https://www.statsmodels.org/dev/generated/statsmodels.tsa.statespace.sarimax.SARIMAX.html

    \paragraph{(Optional) About
seasonality}\label{optional-about-seasonality}

\subparagraph{Whether seasonality effect is
significant:}\label{whether-seasonality-effect-is-significant}

\begin{itemize}
\tightlist
\item
  Observing using your eyes
\item
  ACF plot
\item
  Example of significant seaonality:
\item
  Example 1: 
\item
  Example 2: 
\end{itemize}

\subparagraph{How to remove/control seasonality
effect:}\label{how-to-removecontrol-seasonality-effect}

\begin{itemize}
\tightlist
\item
  By differencing:
  https://www.stat.berkeley.edu/\textasciitilde{}gido/Removal\%20of\%20Trend\%20and\%20Seasonality.pdf
\item
  By adding seasonality dummy variables:
  https://www.ssc.wisc.edu/\textasciitilde{}bhansen/390/390Lecture14.pdf
  and http://faculty.washington.edu/htamura/qm530/cases/seasonality.ppt
\end{itemize}

    \begin{Verbatim}[commandchars=\\\{\}]
{\color{incolor}In [{\color{incolor}14}]:} \PY{c+c1}{\PYZsh{} Evaluating the Residuals for Stationary}
         \PY{n}{data\PYZus{}decompose} \PY{o}{=} \PY{n}{pd}\PY{o}{.}\PY{n}{Series}\PY{p}{(}\PY{n}{residual}\PY{p}{)}
         \PY{n}{data\PYZus{}decompose}\PY{o}{.}\PY{n}{dropna}\PY{p}{(}\PY{n}{inplace}\PY{o}{=}\PY{k+kc}{True}\PY{p}{)}
         
         \PY{n}{t}\PY{o}{=}\PY{l+m+mi}{30} \PY{c+c1}{\PYZsh{} Rolling time window, and calculate mean and standard error}
         \PY{n}{rolmean} \PY{o}{=} \PY{n}{data\PYZus{}decompose}\PY{o}{.}\PY{n}{rolling}\PY{p}{(}\PY{n}{window}\PY{o}{=}\PY{n}{t}\PY{p}{)}\PY{o}{.}\PY{n}{mean}\PY{p}{(}\PY{p}{)}
         \PY{n}{rolstd} \PY{o}{=} \PY{n}{data\PYZus{}decompose}\PY{o}{.}\PY{n}{rolling}\PY{p}{(}\PY{n}{window}\PY{o}{=}\PY{n}{t}\PY{p}{)}\PY{o}{.}\PY{n}{std}\PY{p}{(}\PY{p}{)}
         
         \PY{c+c1}{\PYZsh{} Visualize whether mean and standard error of time series are stationary over time}
         \PY{n}{plt}\PY{o}{.}\PY{n}{figure}\PY{p}{(}\PY{n}{figsize}\PY{o}{=}\PY{p}{(}\PY{l+m+mi}{15}\PY{p}{,}\PY{l+m+mi}{5}\PY{p}{)}\PY{p}{)}
         \PY{n}{orig} \PY{o}{=} \PY{n}{plt}\PY{o}{.}\PY{n}{plot}\PY{p}{(}\PY{n}{data\PYZus{}decompose}\PY{p}{,} \PY{n}{color}\PY{o}{=}\PY{l+s+s1}{\PYZsq{}}\PY{l+s+s1}{blue}\PY{l+s+s1}{\PYZsq{}}\PY{p}{,}\PY{n}{label}\PY{o}{=}\PY{l+s+s1}{\PYZsq{}}\PY{l+s+s1}{Original Residuals}\PY{l+s+s1}{\PYZsq{}}\PY{p}{)}
         \PY{n}{mean} \PY{o}{=} \PY{n}{plt}\PY{o}{.}\PY{n}{plot}\PY{p}{(}\PY{n}{rolmean}\PY{p}{,} \PY{n}{color}\PY{o}{=}\PY{l+s+s1}{\PYZsq{}}\PY{l+s+s1}{red}\PY{l+s+s1}{\PYZsq{}}\PY{p}{,} \PY{n}{label}\PY{o}{=}\PY{l+s+s1}{\PYZsq{}}\PY{l+s+s1}{Rolling Mean}\PY{l+s+s1}{\PYZsq{}}\PY{p}{)}
         \PY{n}{std} \PY{o}{=} \PY{n}{plt}\PY{o}{.}\PY{n}{plot}\PY{p}{(}\PY{n}{rolstd}\PY{p}{,} \PY{n}{color}\PY{o}{=}\PY{l+s+s1}{\PYZsq{}}\PY{l+s+s1}{black}\PY{l+s+s1}{\PYZsq{}}\PY{p}{,} \PY{n}{label} \PY{o}{=} \PY{l+s+s1}{\PYZsq{}}\PY{l+s+s1}{Rolling Std}\PY{l+s+s1}{\PYZsq{}}\PY{p}{)}
         \PY{n}{plt}\PY{o}{.}\PY{n}{legend}\PY{p}{(}\PY{n}{loc}\PY{o}{=}\PY{l+s+s1}{\PYZsq{}}\PY{l+s+s1}{best}\PY{l+s+s1}{\PYZsq{}}\PY{p}{)}
         \PY{n}{plt}\PY{o}{.}\PY{n}{title}\PY{p}{(}\PY{l+s+s1}{\PYZsq{}}\PY{l+s+s1}{Rolling Mean \PYZam{} Standard Deviation of Residuals}\PY{l+s+s1}{\PYZsq{}}\PY{p}{)}
         \PY{n}{plt}\PY{o}{.}\PY{n}{show}\PY{p}{(}\PY{n}{block}\PY{o}{=}\PY{k+kc}{False}\PY{p}{)}
         
         \PY{c+c1}{\PYZsh{} Dicky\PYZhy{}Fuller test on stationary condition}
         \PY{n+nb}{print} \PY{p}{(}\PY{l+s+s1}{\PYZsq{}}\PY{l+s+s1}{Results of Dickey\PYZhy{}Fuller Test:}\PY{l+s+s1}{\PYZsq{}}\PY{p}{)}
         
         \PY{c+c1}{\PYZsh{} Dickey\PYZhy{}Fuller test}
         \PY{n}{dftest} \PY{o}{=} \PY{n}{adfuller}\PY{p}{(}\PY{n}{data\PYZus{}decompose}\PY{p}{,} \PY{n}{autolag}\PY{o}{=}\PY{k+kc}{None}\PY{p}{)} 
         \PY{n}{dfoutput} \PY{o}{=} \PY{n}{pd}\PY{o}{.}\PY{n}{Series}\PY{p}{(}\PY{n}{dftest}\PY{p}{[}\PY{l+m+mi}{0}\PY{p}{:}\PY{l+m+mi}{4}\PY{p}{]}\PY{p}{,} \PY{n}{index}\PY{o}{=}\PY{p}{[}\PY{l+s+s1}{\PYZsq{}}\PY{l+s+s1}{Test Statistic}\PY{l+s+s1}{\PYZsq{}}\PY{p}{,}\PY{l+s+s1}{\PYZsq{}}\PY{l+s+s1}{p\PYZhy{}value}\PY{l+s+s1}{\PYZsq{}}\PY{p}{,}\PY{l+s+s1}{\PYZsq{}}\PY{l+s+s1}{\PYZsh{}Lags Used}\PY{l+s+s1}{\PYZsq{}}\PY{p}{,}\PY{l+s+s1}{\PYZsq{}}\PY{l+s+s1}{Number of Observations Used}\PY{l+s+s1}{\PYZsq{}}\PY{p}{]}\PY{p}{)}
         
         \PY{k}{for} \PY{n}{key}\PY{p}{,} \PY{n}{value} \PY{o+ow}{in} \PY{n}{dftest}\PY{p}{[}\PY{l+m+mi}{4}\PY{p}{]}\PY{o}{.}\PY{n}{items}\PY{p}{(}\PY{p}{)}\PY{p}{:}
             \PY{n}{dfoutput}\PY{p}{[}\PY{l+s+s1}{\PYZsq{}}\PY{l+s+s1}{Critical Value (}\PY{l+s+si}{\PYZpc{}s}\PY{l+s+s1}{)}\PY{l+s+s1}{\PYZsq{}}\PY{o}{\PYZpc{}}\PY{k}{key}] = value
         
         \PY{n+nb}{print} \PY{p}{(}\PY{n}{dfoutput}\PY{p}{)}
\end{Verbatim}


    \begin{center}
    \adjustimage{max size={0.9\linewidth}{0.9\paperheight}}{output_31_0.png}
    \end{center}
    { \hspace*{\fill} \\}
    
    \begin{Verbatim}[commandchars=\\\{\}]
Results of Dickey-Fuller Test:
Test Statistic                  -4.801441
p-value                          0.000054
\#Lags Used                      14.000000
Number of Observations Used    147.000000
Critical Value (1\%)             -3.475637
Critical Value (5\%)             -2.881410
Critical Value (10\%)            -2.577365
dtype: float64

    \end{Verbatim}

    \paragraph{The residuals are stationary, but residuals may not seem like
random noise (i.e., standard normal distribution). This is not a big
issue. As long as the series become stationary now, you can fit the time
series with time series model, such as ARMA or ARIMA. If you verify that
variance may change within short-term periods, you can use models such
as ARCH or GARCH to fit such
residuals.}\label{the-residuals-are-stationary-but-residuals-may-not-seem-like-random-noise-i.e.-standard-normal-distribution.-this-is-not-a-big-issue.-as-long-as-the-series-become-stationary-now-you-can-fit-the-time-series-with-time-series-model-such-as-arma-or-arima.-if-you-verify-that-variance-may-change-within-short-term-periods-you-can-use-models-such-as-arch-or-garch-to-fit-such-residuals.}

\paragraph{If you want to know whether your stationary residual series
have autocorrelation issue, you can visualize it using Autocorrelation
Function (ACF), or check with statistical tests (In tutorial
5).}\label{if-you-want-to-know-whether-your-stationary-residual-series-have-autocorrelation-issue-you-can-visualize-it-using-autocorrelation-function-acf-or-check-with-statistical-tests-in-tutorial-5.}

    \paragraph{2.3.4 (Optional) Autocorrelation Function and Partial
Autocorrelation
Function}\label{optional-autocorrelation-function-and-partial-autocorrelation-function}

Autocorrelation function (ACF) determines the correlation of time series
along with a lagged version of itself. Partial autocorrelation function
(PACF) also measures the correlation of a time series along with a
lagged version of itself except that it eliminates the variations
already explained by the prior comparisons. You may check the references
and see the differences between them.

You can also use Durbin-Watson Test and Portmanteau Test to verify
whether autocorrelation exists.

References: * https://onlinecourses.science.psu.edu/stat510/node/62/ *
https://onlinecourses.science.psu.edu/stat510/node/60/ * Why we say
partial correlation is very similar to regression coefficients:
http://www.ernestoamaral.com/docs/soci420-17fall/Lecture24.pdf * ACF and
PACF:
https://machinelearningmastery.com/gentle-introduction-autocorrelation-partial-autocorrelation/

    \begin{Verbatim}[commandchars=\\\{\}]
{\color{incolor}In [{\color{incolor}17}]:} \PY{c+c1}{\PYZsh{} Plotting Correlograms for ACF/PACF on Residuals}
         \PY{n}{plt}\PY{o}{.}\PY{n}{figure}\PY{p}{(}\PY{n}{figsize}\PY{o}{=}\PY{p}{(}\PY{l+m+mi}{15}\PY{p}{,}\PY{l+m+mi}{10}\PY{p}{)}\PY{p}{)}
         \PY{n}{ax1} \PY{o}{=} \PY{n}{plt}\PY{o}{.}\PY{n}{subplot}\PY{p}{(}\PY{l+m+mi}{211}\PY{p}{)}
         \PY{n}{fig} \PY{o}{=} \PY{n}{sm}\PY{o}{.}\PY{n}{graphics}\PY{o}{.}\PY{n}{tsa}\PY{o}{.}\PY{n}{plot\PYZus{}acf}\PY{p}{(}\PY{n}{data\PYZus{}decompose}\PY{o}{.}\PY{n}{squeeze}\PY{p}{(}\PY{p}{)}\PY{p}{,} \PY{n}{lags}\PY{o}{=}\PY{l+m+mi}{40}\PY{p}{,} \PY{n}{ax}\PY{o}{=}\PY{n}{ax1}\PY{p}{)}
         \PY{c+c1}{\PYZsh{}ax2 = plt.subplot(212)}
         \PY{c+c1}{\PYZsh{}fig = sm.graphics.tsa.plot\PYZus{}pacf(data\PYZus{}decompose, lags=40, ax=ax2)}
\end{Verbatim}


    \begin{center}
    \adjustimage{max size={0.9\linewidth}{0.9\paperheight}}{output_34_0.png}
    \end{center}
    { \hspace*{\fill} \\}
    
    \paragraph{Only the first-order autocorrelation may be significant
(i.e., correlation of current residual and lag-1 of residual is
high).}\label{only-the-first-order-autocorrelation-may-be-significant-i.e.-correlation-of-current-residual-and-lag-1-of-residual-is-high.}

    \paragraph{2.3.5 Time Series Forecasting with Autoregression and Moving
Average
(ARMA)}\label{time-series-forecasting-with-autoregression-and-moving-average-arma}

\paragraph{After you transform your original non-stationary time series
to stationary time series, you can fit ARMA or ARIMA model on this
stationary time
series.}\label{after-you-transform-your-original-non-stationary-time-series-to-stationary-time-series-you-can-fit-arma-or-arima-model-on-this-stationary-time-series.}

\textbf{Autoregression Intuition} * Consider a time series that was
generated by an autoregression (AR) process with a lag of k. * We know
that the ACF describes the autocorrelation between an observation and
another observation at a prior time step that includes direct and
indirect dependence information. * This means we would expect the ACF
for the AR(k) time series to be strong to a lag of k and the inertia of
that relationship would carry on to subsequent lag values, trailing off
at some point as the effect was weakened. * We know that the PACF only
describes the direct relationship between an observation and its lag.
This would suggest that there would be no correlation for lag values
beyond k. * This is exactly the expectation of the ACF and PACF plots
for an AR(k) process.

\textbf{Moving Average Intuition} * Consider a time series that was
generated by a moving average (MA) process with a lag of k. * Remember
that the moving average process is an autoregression model of the time
series of residual errors from prior predictions. Another way to think
about the moving average model is that it corrects future forecasts
based on errors made on recent forecasts. * We would expect the ACF for
the MA(k) process to show a strong correlation with recent values up to
the lag of k, then a sharp decline to low or no correlation. By
definition, this is how the process was generated. * For the PACF, we
would expect the plot to show a strong relationship to the lag and a
trailing off of correlation from the lag onwards. * Again, this is
exactly the expectation of the ACF and PACF plots for an MA(k) process.

References: * Why we say partial correlation is very similar to
regression coefficients:
http://www.ernestoamaral.com/docs/soci420-17fall/Lecture24.pdf * ACF and
PACF:
https://machinelearningmastery.com/gentle-introduction-autocorrelation-partial-autocorrelation/
* Moving Average:
https://mcs.utm.utoronto.ca/\textasciitilde{}nosedal/sta457/ma1-and-ma2.pdf
and
https://datamfr.files.wordpress.com/2012/10/acf-and-pacf-arima-models.pdf
* \texttt{statsmodels}:
https://www.statsmodels.org/dev/generated/statsmodels.tsa.arima\_model.ARIMA.html
and
https://www.statsmodels.org/dev/generated/statsmodels.tsa.arima\_model.ARIMA.fit.html\#statsmodels.tsa.arima\_model.ARIMA.fit

    \paragraph{ARIMA model is an ARMA model allowing for differencing on the
original data. In this example, we directly fit an ARIMA model on the
original non-stationary time
series.}\label{arima-model-is-an-arma-model-allowing-for-differencing-on-the-original-data.-in-this-example-we-directly-fit-an-arima-model-on-the-original-non-stationary-time-series.}

\paragraph{Remember that in this case, after first-differencing on the
original time series, we get a stationary series. So we can directly
apply ARIMA model on the original
data.}\label{remember-that-in-this-case-after-first-differencing-on-the-original-time-series-we-get-a-stationary-series.-so-we-can-directly-apply-arima-model-on-the-original-data.}

References:
https://en.wikipedia.org/wiki/Autoregressive\_integrated\_moving\_average

    \begin{Verbatim}[commandchars=\\\{\}]
{\color{incolor}In [{\color{incolor}18}]:} \PY{c+c1}{\PYZsh{} Suppose we use ARIMA(p,d,q) to fit the time series data: AR(p), MA(q) and d\PYZhy{}order\PYZhy{}differencing}
         \PY{c+c1}{\PYZsh{} Remember: we have tested that first\PYZhy{}order\PYZhy{}differencing is effective enough to make non\PYZhy{}stationary to stationary }
         \PY{c+c1}{\PYZsh{} Here we can use ARIMA model, which fits ARMA model on (first\PYZhy{}order) differencing time series data}
         \PY{k+kn}{from} \PY{n+nn}{statsmodels}\PY{n+nn}{.}\PY{n+nn}{tsa}\PY{n+nn}{.}\PY{n+nn}{arima\PYZus{}model} \PY{k}{import} \PY{n}{ARIMA}
         \PY{k+kn}{import} \PY{n+nn}{seaborn} \PY{k}{as} \PY{n+nn}{sns}
         
         \PY{c+c1}{\PYZsh{} ARIMA(5, 1, 5): AR(5), MA(5) and first\PYZhy{}order\PYZhy{}differencing}
         \PY{n}{model} \PY{o}{=} \PY{n}{ARIMA}\PY{p}{(}\PY{n}{data\PYZus{}train}\PY{p}{[}\PY{l+s+s1}{\PYZsq{}}\PY{l+s+s1}{Adj Close}\PY{l+s+s1}{\PYZsq{}}\PY{p}{]}\PY{p}{,} \PY{n}{order}\PY{o}{=}\PY{p}{(}\PY{l+m+mi}{5}\PY{p}{,}\PY{l+m+mi}{1}\PY{p}{,}\PY{l+m+mi}{5}\PY{p}{)}\PY{p}{)}
         \PY{n}{model\PYZus{}fit} \PY{o}{=} \PY{n}{model}\PY{o}{.}\PY{n}{fit}\PY{p}{(}\PY{n}{disp}\PY{o}{=}\PY{l+m+mi}{0}\PY{p}{)}
         \PY{n+nb}{print} \PY{p}{(}\PY{n}{model\PYZus{}fit}\PY{o}{.}\PY{n}{summary}\PY{p}{(}\PY{p}{)}\PY{p}{)}
         \PY{c+c1}{\PYZsh{} Find AIC value}
         \PY{n+nb}{print} \PY{p}{(}\PY{l+s+s2}{\PYZdq{}}\PY{l+s+s2}{AIC is: }\PY{l+s+si}{\PYZpc{}s}\PY{l+s+s2}{\PYZdq{}} \PY{o}{\PYZpc{}}\PY{p}{(}\PY{n}{model\PYZus{}fit}\PY{o}{.}\PY{n}{aic}\PY{p}{)}\PY{p}{)}
         
         \PY{c+c1}{\PYZsh{} Plot residual errors series}
         \PY{n}{plt}\PY{o}{.}\PY{n}{figure}\PY{p}{(}\PY{p}{)}
         \PY{n}{residuals} \PY{o}{=} \PY{n}{pd}\PY{o}{.}\PY{n}{DataFrame}\PY{p}{(}\PY{n}{model\PYZus{}fit}\PY{o}{.}\PY{n}{resid}\PY{p}{)}
         \PY{n}{residuals}\PY{o}{.}\PY{n}{plot}\PY{p}{(}\PY{p}{)}
         \PY{n}{plt}\PY{o}{.}\PY{n}{show}\PY{p}{(}\PY{p}{)}
         
         \PY{c+c1}{\PYZsh{} Plot distribution of residual errors}
         \PY{n}{sns}\PY{o}{.}\PY{n}{distplot}\PY{p}{(}\PY{n}{residuals}\PY{p}{,} \PY{n}{hist}\PY{o}{=}\PY{k+kc}{True}\PY{p}{,} \PY{n}{kde}\PY{o}{=}\PY{k+kc}{True}\PY{p}{,} \PY{n}{bins}\PY{o}{=}\PY{l+m+mi}{50}\PY{p}{,} \PY{n}{color} \PY{o}{=} \PY{l+s+s1}{\PYZsq{}}\PY{l+s+s1}{darkblue}\PY{l+s+s1}{\PYZsq{}}\PY{p}{,} \PY{n}{hist\PYZus{}kws}\PY{o}{=}\PY{p}{\PYZob{}}\PY{l+s+s1}{\PYZsq{}}\PY{l+s+s1}{edgecolor}\PY{l+s+s1}{\PYZsq{}}\PY{p}{:}\PY{l+s+s1}{\PYZsq{}}\PY{l+s+s1}{black}\PY{l+s+s1}{\PYZsq{}}\PY{p}{\PYZcb{}}\PY{p}{,} \PY{n}{kde\PYZus{}kws}\PY{o}{=}\PY{p}{\PYZob{}}\PY{l+s+s1}{\PYZsq{}}\PY{l+s+s1}{linewidth}\PY{l+s+s1}{\PYZsq{}}\PY{p}{:} \PY{l+m+mi}{4}\PY{p}{\PYZcb{}}\PY{p}{)}
\end{Verbatim}


    \begin{Verbatim}[commandchars=\\\{\}]
                             ARIMA Model Results                              
==============================================================================
Dep. Variable:            D.Adj Close   No. Observations:                  175
Model:                 ARIMA(5, 1, 5)   Log Likelihood                -657.330
Method:                       css-mle   S.D. of innovations             10.135
Date:                Fri, 28 Sep 2018   AIC                           1338.661
Time:                        13:13:35   BIC                           1376.638
Sample:                             1   HQIC                          1354.066
                                                                              
=====================================================================================
                        coef    std err          z      P>|z|      [0.025      0.975]
-------------------------------------------------------------------------------------
const                 1.2173      0.137      8.897      0.000       0.949       1.485
ar.L1.D.Adj Close     0.8832      0.097      9.103      0.000       0.693       1.073
ar.L2.D.Adj Close    -0.2220      0.160     -1.391      0.166      -0.535       0.091
ar.L3.D.Adj Close     0.2851      0.110      2.583      0.011       0.069       0.501
ar.L4.D.Adj Close    -0.9618      0.049    -19.657      0.000      -1.058      -0.866
ar.L5.D.Adj Close     0.8194      0.066     12.357      0.000       0.689       0.949
ma.L1.D.Adj Close    -1.0716      0.077    -13.903      0.000      -1.223      -0.921
ma.L2.D.Adj Close     0.2889      0.144      2.003      0.047       0.006       0.572
ma.L3.D.Adj Close    -0.2889      0.141     -2.055      0.041      -0.564      -0.013
ma.L4.D.Adj Close     1.0716      0.077     13.993      0.000       0.922       1.222
ma.L5.D.Adj Close    -1.0000      0.064    -15.590      0.000      -1.126      -0.874
                                    Roots                                    
=============================================================================
                  Real          Imaginary           Modulus         Frequency
-----------------------------------------------------------------------------
AR.1           -0.6599           -0.7615j            1.0077           -0.3636
AR.2           -0.6599           +0.7615j            1.0077            0.3636
AR.3            0.6955           -0.7786j            1.0440           -0.1340
AR.4            0.6955           +0.7786j            1.0440            0.1340
AR.5            1.1027           -0.0000j            1.1027           -0.0000
MA.1           -0.6499           -0.7600j            1.0000           -0.3626
MA.2           -0.6499           +0.7600j            1.0000            0.3626
MA.3            0.6857           -0.7278j            1.0000           -0.1297
MA.4            0.6857           +0.7278j            1.0000            0.1297
MA.5            1.0000           -0.0000j            1.0000           -0.0000
-----------------------------------------------------------------------------
AIC is: 1338.6609969483375

    \end{Verbatim}

    
    \begin{verbatim}
<Figure size 432x288 with 0 Axes>
    \end{verbatim}

    
    \begin{center}
    \adjustimage{max size={0.9\linewidth}{0.9\paperheight}}{output_38_2.png}
    \end{center}
    { \hspace*{\fill} \\}
    
\begin{Verbatim}[commandchars=\\\{\}]
{\color{outcolor}Out[{\color{outcolor}18}]:} <matplotlib.axes.\_subplots.AxesSubplot at 0x1e65a629d68>
\end{Verbatim}
            
    \begin{center}
    \adjustimage{max size={0.9\linewidth}{0.9\paperheight}}{output_38_4.png}
    \end{center}
    { \hspace*{\fill} \\}
    
    \paragraph{Sometimes you may get the error message: "The computed
initial AR coefficients are not stationary You should induce
stationarity, choose a different model order, or you can pass your own
start\_params." It means after de-trending, your time series data may
still be non-stationary (i.e., seasonality may have significant effect).
Then you can try SARIMA (seasonal ARIMA) model to fit the
data.}\label{sometimes-you-may-get-the-error-message-the-computed-initial-ar-coefficients-are-not-stationary-you-should-induce-stationarity-choose-a-different-model-order-or-you-can-pass-your-own-start_params.-it-means-after-de-trending-your-time-series-data-may-still-be-non-stationary-i.e.-seasonality-may-have-significant-effect.-then-you-can-try-sarima-seasonal-arima-model-to-fit-the-data.}

    \begin{Verbatim}[commandchars=\\\{\}]
{\color{incolor}In [{\color{incolor}19}]:} \PY{c+c1}{\PYZsh{} We can do a grid search on selecting autoregressive order or moving average order in ARIMA model}
         \PY{c+c1}{\PYZsh{} Select the model with best performance}
         \PY{c+c1}{\PYZsh{} Remember usually you need to further split training data to do validation, and find the model with best validation accuracy. }
         \PY{c+c1}{\PYZsh{} In this example, in order to simply and quickly illustrate the procedure of time series modelling, we ignore this step.}
         \PY{n}{p\PYZus{}list} \PY{o}{=} \PY{p}{[}\PY{n}{i} \PY{k}{for} \PY{n}{i} \PY{o+ow}{in} \PY{n+nb}{range}\PY{p}{(}\PY{l+m+mi}{1}\PY{p}{,} \PY{l+m+mi}{11}\PY{p}{)}\PY{p}{]} \PY{c+c1}{\PYZsh{} AR order}
         \PY{n}{best\PYZus{}AIC} \PY{o}{=} \PY{n+nb}{float}\PY{p}{(}\PY{l+s+s1}{\PYZsq{}}\PY{l+s+s1}{inf}\PY{l+s+s1}{\PYZsq{}}\PY{p}{)} \PY{c+c1}{\PYZsh{} You need to minimize AIC}
         \PY{n}{best\PYZus{}model} \PY{o}{=} \PY{k+kc}{None} \PY{c+c1}{\PYZsh{} Model with lowest AIC}
         
         \PY{k}{for} \PY{n}{p} \PY{o+ow}{in} \PY{n}{p\PYZus{}list}\PY{p}{:}
             \PY{c+c1}{\PYZsh{} ARIMA(p, 1, 1): AR(p), MA(1) and first\PYZhy{}order\PYZhy{}differencing}
             \PY{n}{model} \PY{o}{=} \PY{n}{ARIMA}\PY{p}{(}\PY{n}{data\PYZus{}train}\PY{p}{[}\PY{l+s+s1}{\PYZsq{}}\PY{l+s+s1}{Adj Close}\PY{l+s+s1}{\PYZsq{}}\PY{p}{]}\PY{p}{,} \PY{n}{order}\PY{o}{=}\PY{p}{(}\PY{n}{p}\PY{p}{,}\PY{l+m+mi}{1}\PY{p}{,}\PY{l+m+mi}{1}\PY{p}{)}\PY{p}{)}
             \PY{n}{model\PYZus{}fit} \PY{o}{=} \PY{n}{model}\PY{o}{.}\PY{n}{fit}\PY{p}{(}\PY{n}{disp}\PY{o}{=}\PY{l+m+mi}{0}\PY{p}{)}
                 
             \PY{k}{if} \PY{n}{model\PYZus{}fit}\PY{o}{.}\PY{n}{aic} \PY{o}{\PYZlt{}}\PY{o}{=} \PY{n}{best\PYZus{}AIC}\PY{p}{:}
                 \PY{n}{best\PYZus{}model}\PY{p}{,} \PY{n}{best\PYZus{}AIC} \PY{o}{=} \PY{n}{model}\PY{p}{,} \PY{n}{model\PYZus{}fit}\PY{o}{.}\PY{n}{aic}
\end{Verbatim}


    \paragraph{Sometimes you may get the error message: "The computed
initial AR coefficients are not stationary You should induce
stationarity, choose a different model order, or you can pass your own
start\_params." It means after de-trending, your time series data may
still be non-stationary (i.e., seasonality may have significant effect).
Then you can try SARIMA (seasonal ARIMA) model to fit the
data.}\label{sometimes-you-may-get-the-error-message-the-computed-initial-ar-coefficients-are-not-stationary-you-should-induce-stationarity-choose-a-different-model-order-or-you-can-pass-your-own-start_params.-it-means-after-de-trending-your-time-series-data-may-still-be-non-stationary-i.e.-seasonality-may-have-significant-effect.-then-you-can-try-sarima-seasonal-arima-model-to-fit-the-data.}

    \subsubsection{In-sample forecasting: Whether model fits training data
well}\label{in-sample-forecasting-whether-model-fits-training-data-well}

    \begin{Verbatim}[commandchars=\\\{\}]
{\color{incolor}In [{\color{incolor}20}]:} \PY{n}{best\PYZus{}model\PYZus{}fit} \PY{o}{=} \PY{n}{best\PYZus{}model}\PY{o}{.}\PY{n}{fit}\PY{p}{(}\PY{n}{disp}\PY{o}{=}\PY{l+m+mi}{0}\PY{p}{)}
         \PY{n}{best\PYZus{}model\PYZus{}fit}\PY{o}{.}\PY{n}{summary}\PY{p}{(}\PY{p}{)}
         \PY{n}{best\PYZus{}model\PYZus{}fit}\PY{o}{.}\PY{n}{plot\PYZus{}predict}\PY{p}{(}\PY{p}{)} \PY{c+c1}{\PYZsh{} Visualize in\PYZhy{}sample forecasting: Whether model fits training data well}
         \PY{c+c1}{\PYZsh{} best\PYZus{}model\PYZus{}fit.predict(typ=\PYZsq{}levels\PYZsq{}) \PYZsh{} In\PYZhy{}sample fitted values}
\end{Verbatim}

\texttt{\color{outcolor}Out[{\color{outcolor}20}]:}
    
    \begin{center}
    \adjustimage{max size={0.9\linewidth}{0.9\paperheight}}{output_43_0.png}
    \end{center}
    { \hspace*{\fill} \\}
    

    \begin{center}
    \adjustimage{max size={0.9\linewidth}{0.9\paperheight}}{output_43_1.png}
    \end{center}
    { \hspace*{\fill} \\}
    
    \subsubsection{Out-of-sample Forecasting: Prediction using test data:
calculate RMSE on test
data}\label{out-of-sample-forecasting-prediction-using-test-data-calculate-rmse-on-test-data}

    \begin{Verbatim}[commandchars=\\\{\}]
{\color{incolor}In [{\color{incolor}21}]:} \PY{c+c1}{\PYZsh{} Out\PYZhy{}of\PYZhy{}sample Forcasting: Prediction using test data: calculate RMSE on test data}
         \PY{c+c1}{\PYZsh{} https://www.statsmodels.org/dev/generated/statsmodels.tsa.arima\PYZus{}model.ARMAResults.forecast.html}
         \PY{k+kn}{from} \PY{n+nn}{copy} \PY{k}{import} \PY{n}{deepcopy}
         \PY{n}{y\PYZus{}hat\PYZus{}test} \PY{o}{=} \PY{n}{deepcopy}\PY{p}{(}\PY{n}{data\PYZus{}test}\PY{p}{)}
         \PY{n}{y\PYZus{}hat\PYZus{}test}\PY{p}{[}\PY{l+s+s1}{\PYZsq{}}\PY{l+s+s1}{predict}\PY{l+s+s1}{\PYZsq{}}\PY{p}{]} \PY{o}{=} \PY{n}{best\PYZus{}model\PYZus{}fit}\PY{o}{.}\PY{n}{forecast}\PY{p}{(}\PY{n+nb}{len}\PY{p}{(}\PY{n}{data\PYZus{}test}\PY{p}{)}\PY{p}{)}\PY{p}{[}\PY{l+m+mi}{0}\PY{p}{]}
         
         \PY{c+c1}{\PYZsh{} Plot}
         \PY{n}{plt}\PY{o}{.}\PY{n}{figure}\PY{p}{(}\PY{n}{figsize}\PY{o}{=}\PY{p}{(}\PY{l+m+mi}{15}\PY{p}{,}\PY{l+m+mi}{5}\PY{p}{)}\PY{p}{)}
         \PY{n}{plt}\PY{o}{.}\PY{n}{plot}\PY{p}{(}\PY{n}{data\PYZus{}train}\PY{p}{[}\PY{l+s+s1}{\PYZsq{}}\PY{l+s+s1}{Adj Close}\PY{l+s+s1}{\PYZsq{}}\PY{p}{]}\PY{p}{,} \PY{n}{label}\PY{o}{=}\PY{l+s+s1}{\PYZsq{}}\PY{l+s+s1}{Train}\PY{l+s+s1}{\PYZsq{}}\PY{p}{)}
         \PY{n}{plt}\PY{o}{.}\PY{n}{plot}\PY{p}{(}\PY{n}{data\PYZus{}test}\PY{p}{[}\PY{l+s+s1}{\PYZsq{}}\PY{l+s+s1}{Adj Close}\PY{l+s+s1}{\PYZsq{}}\PY{p}{]}\PY{p}{,} \PY{n}{label}\PY{o}{=}\PY{l+s+s1}{\PYZsq{}}\PY{l+s+s1}{Test}\PY{l+s+s1}{\PYZsq{}}\PY{p}{)}
         \PY{n}{plt}\PY{o}{.}\PY{n}{plot}\PY{p}{(}\PY{n}{y\PYZus{}hat\PYZus{}test}\PY{p}{[}\PY{l+s+s1}{\PYZsq{}}\PY{l+s+s1}{predict}\PY{l+s+s1}{\PYZsq{}}\PY{p}{]}\PY{p}{,} \PY{n}{label}\PY{o}{=}\PY{l+s+s1}{\PYZsq{}}\PY{l+s+s1}{Predict}\PY{l+s+s1}{\PYZsq{}}\PY{p}{)}
         \PY{n}{plt}\PY{o}{.}\PY{n}{title}\PY{p}{(}\PY{l+s+s1}{\PYZsq{}}\PY{l+s+s1}{Forecasting RMSE: }\PY{l+s+si}{\PYZpc{}.4f}\PY{l+s+s1}{\PYZsq{}}\PY{o}{\PYZpc{}} \PY{n}{np}\PY{o}{.}\PY{n}{sqrt}\PY{p}{(}\PY{n+nb}{sum}\PY{p}{(}\PY{p}{(}\PY{n}{y\PYZus{}hat\PYZus{}test}\PY{p}{[}\PY{l+s+s1}{\PYZsq{}}\PY{l+s+s1}{predict}\PY{l+s+s1}{\PYZsq{}}\PY{p}{]}\PY{o}{\PYZhy{}}\PY{n}{y\PYZus{}hat\PYZus{}test}\PY{p}{[}\PY{l+s+s1}{\PYZsq{}}\PY{l+s+s1}{Adj Close}\PY{l+s+s1}{\PYZsq{}}\PY{p}{]}\PY{p}{)}\PY{o}{*}\PY{o}{*}\PY{l+m+mi}{2}\PY{p}{)}\PY{o}{/}\PY{n+nb}{len}\PY{p}{(}\PY{n}{y\PYZus{}hat\PYZus{}test}\PY{p}{)}\PY{p}{)}\PY{p}{)}
         \PY{n}{plt}\PY{o}{.}\PY{n}{legend}\PY{p}{(}\PY{n}{loc}\PY{o}{=}\PY{l+s+s1}{\PYZsq{}}\PY{l+s+s1}{best}\PY{l+s+s1}{\PYZsq{}}\PY{p}{)}
         \PY{n}{plt}\PY{o}{.}\PY{n}{show}\PY{p}{(}\PY{p}{)}
\end{Verbatim}


    \begin{center}
    \adjustimage{max size={0.9\linewidth}{0.9\paperheight}}{output_45_0.png}
    \end{center}
    { \hspace*{\fill} \\}
    
    \begin{Verbatim}[commandchars=\\\{\}]
{\color{incolor}In [{\color{incolor}22}]:} \PY{c+c1}{\PYZsh{} Rolling forecast using test data: calculate RMSE on test data}
         \PY{c+c1}{\PYZsh{} A rolling forecast is required given the dependence on observations in prior time steps for differencing and the AR model. }
         \PY{c+c1}{\PYZsh{} A crude way to perform this rolling forecast is to re\PYZhy{}create the ARIMA model after each new observation is received.}
         \PY{c+c1}{\PYZsh{} https://robjhyndman.com/hyndsight/tscv/}
         \PY{c+c1}{\PYZsh{} https://www.statsmodels.org/dev/generated/statsmodels.tsa.arima\PYZus{}model.ARMAResults.forecast.html}
         \PY{k+kn}{from} \PY{n+nn}{copy} \PY{k}{import} \PY{n}{deepcopy}
         \PY{n}{y\PYZus{}hat\PYZus{}test} \PY{o}{=} \PY{n}{deepcopy}\PY{p}{(}\PY{n}{data\PYZus{}test}\PY{p}{)}
         
         \PY{n}{history} \PY{o}{=} \PY{p}{[}\PY{n}{x} \PY{k}{for} \PY{n}{x} \PY{o+ow}{in} \PY{n}{deepcopy}\PY{p}{(}\PY{n}{data\PYZus{}train}\PY{p}{[}\PY{l+s+s1}{\PYZsq{}}\PY{l+s+s1}{Adj Close}\PY{l+s+s1}{\PYZsq{}}\PY{p}{]}\PY{p}{)}\PY{p}{]}
         \PY{n}{predictions} \PY{o}{=} \PY{n+nb}{list}\PY{p}{(}\PY{p}{)}
         \PY{k}{for} \PY{n}{t} \PY{o+ow}{in} \PY{n+nb}{range}\PY{p}{(}\PY{n+nb}{len}\PY{p}{(}\PY{n}{y\PYZus{}hat\PYZus{}test}\PY{p}{)}\PY{p}{)}\PY{p}{:}
             \PY{n}{model} \PY{o}{=} \PY{n}{ARIMA}\PY{p}{(}\PY{n}{history}\PY{p}{,} \PY{n}{order}\PY{o}{=}\PY{p}{(}\PY{l+m+mi}{5}\PY{p}{,}\PY{l+m+mi}{1}\PY{p}{,}\PY{l+m+mi}{0}\PY{p}{)}\PY{p}{)}
             \PY{n}{model\PYZus{}fit} \PY{o}{=} \PY{n}{model}\PY{o}{.}\PY{n}{fit}\PY{p}{(}\PY{n}{disp}\PY{o}{=}\PY{l+m+mi}{0}\PY{p}{)}
             \PY{n}{output} \PY{o}{=} \PY{n}{model\PYZus{}fit}\PY{o}{.}\PY{n}{forecast}\PY{p}{(}\PY{p}{)} \PY{c+c1}{\PYZsh{} Forecast the next instance}
             \PY{n}{yhat} \PY{o}{=} \PY{n}{output}\PY{p}{[}\PY{l+m+mi}{0}\PY{p}{]}\PY{p}{[}\PY{l+m+mi}{0}\PY{p}{]}
             \PY{n}{predictions}\PY{o}{.}\PY{n}{append}\PY{p}{(}\PY{n}{yhat}\PY{p}{)}
             \PY{n}{obs} \PY{o}{=} \PY{n}{y\PYZus{}hat\PYZus{}test}\PY{o}{.}\PY{n}{iloc}\PY{p}{[}\PY{n}{t}\PY{p}{,}\PY{l+m+mi}{1}\PY{p}{]}
             \PY{c+c1}{\PYZsh{} first = history.pop(0) \PYZsh{} You can moving the training time window to keep time window fixed}
             \PY{n}{history}\PY{o}{.}\PY{n}{append}\PY{p}{(}\PY{n}{obs}\PY{p}{)} \PY{c+c1}{\PYZsh{} Update the training data to forecast the next instance}
         
         \PY{n}{y\PYZus{}hat\PYZus{}test}\PY{p}{[}\PY{l+s+s1}{\PYZsq{}}\PY{l+s+s1}{predict}\PY{l+s+s1}{\PYZsq{}}\PY{p}{]} \PY{o}{=} \PY{n}{predictions}
             
         \PY{c+c1}{\PYZsh{} Plot}
         \PY{n}{plt}\PY{o}{.}\PY{n}{figure}\PY{p}{(}\PY{n}{figsize}\PY{o}{=}\PY{p}{(}\PY{l+m+mi}{15}\PY{p}{,}\PY{l+m+mi}{5}\PY{p}{)}\PY{p}{)}
         \PY{n}{plt}\PY{o}{.}\PY{n}{plot}\PY{p}{(}\PY{n}{data\PYZus{}train}\PY{p}{[}\PY{l+s+s1}{\PYZsq{}}\PY{l+s+s1}{Adj Close}\PY{l+s+s1}{\PYZsq{}}\PY{p}{]}\PY{p}{,} \PY{n}{label}\PY{o}{=}\PY{l+s+s1}{\PYZsq{}}\PY{l+s+s1}{Train}\PY{l+s+s1}{\PYZsq{}}\PY{p}{)}
         \PY{n}{plt}\PY{o}{.}\PY{n}{plot}\PY{p}{(}\PY{n}{data\PYZus{}test}\PY{p}{[}\PY{l+s+s1}{\PYZsq{}}\PY{l+s+s1}{Adj Close}\PY{l+s+s1}{\PYZsq{}}\PY{p}{]}\PY{p}{,} \PY{n}{label}\PY{o}{=}\PY{l+s+s1}{\PYZsq{}}\PY{l+s+s1}{Test}\PY{l+s+s1}{\PYZsq{}}\PY{p}{)}
         \PY{n}{plt}\PY{o}{.}\PY{n}{plot}\PY{p}{(}\PY{n}{y\PYZus{}hat\PYZus{}test}\PY{p}{[}\PY{l+s+s1}{\PYZsq{}}\PY{l+s+s1}{predict}\PY{l+s+s1}{\PYZsq{}}\PY{p}{]}\PY{p}{,} \PY{n}{label}\PY{o}{=}\PY{l+s+s1}{\PYZsq{}}\PY{l+s+s1}{Predict}\PY{l+s+s1}{\PYZsq{}}\PY{p}{)}
         \PY{n}{plt}\PY{o}{.}\PY{n}{title}\PY{p}{(}\PY{l+s+s1}{\PYZsq{}}\PY{l+s+s1}{Rolling Forecasting RMSE: }\PY{l+s+si}{\PYZpc{}.4f}\PY{l+s+s1}{\PYZsq{}}\PY{o}{\PYZpc{}} \PY{n}{np}\PY{o}{.}\PY{n}{sqrt}\PY{p}{(}\PY{n+nb}{sum}\PY{p}{(}\PY{p}{(}\PY{n}{y\PYZus{}hat\PYZus{}test}\PY{p}{[}\PY{l+s+s1}{\PYZsq{}}\PY{l+s+s1}{predict}\PY{l+s+s1}{\PYZsq{}}\PY{p}{]}\PY{o}{\PYZhy{}}\PY{n}{y\PYZus{}hat\PYZus{}test}\PY{p}{[}\PY{l+s+s1}{\PYZsq{}}\PY{l+s+s1}{Adj Close}\PY{l+s+s1}{\PYZsq{}}\PY{p}{]}\PY{p}{)}\PY{o}{*}\PY{o}{*}\PY{l+m+mi}{2}\PY{p}{)}\PY{o}{/}\PY{n+nb}{len}\PY{p}{(}\PY{n}{y\PYZus{}hat\PYZus{}test}\PY{p}{)}\PY{p}{)}\PY{p}{)}
         \PY{n}{plt}\PY{o}{.}\PY{n}{legend}\PY{p}{(}\PY{n}{loc}\PY{o}{=}\PY{l+s+s1}{\PYZsq{}}\PY{l+s+s1}{best}\PY{l+s+s1}{\PYZsq{}}\PY{p}{)}
         \PY{n}{plt}\PY{o}{.}\PY{n}{show}\PY{p}{(}\PY{p}{)}
\end{Verbatim}


    \begin{center}
    \adjustimage{max size={0.9\linewidth}{0.9\paperheight}}{output_46_0.png}
    \end{center}
    { \hspace*{\fill} \\}
    
    \paragraph{You can see that (k-step) rolling forecasting gives better
prediction accuracy. In fact, you can view rolling forecasting as
"cross-validation" of time series
data.}\label{you-can-see-that-k-step-rolling-forecasting-gives-better-prediction-accuracy.-in-fact-you-can-view-rolling-forecasting-as-cross-validation-of-time-series-data.}

References: https://robjhyndman.com/hyndsight/tscv/

    \paragraph{2.3.6 (Optional) Alternative Time Series
Models}\label{optional-alternative-time-series-models}

\begin{itemize}
\tightlist
\item
  \textbf{SARIMA} (Seasonal Autoregressive Integrated Moving Average):
  As introduced above, this method addresses seasonality issue when
  seasonablity is a significant reason for non-stationarity.
\item
  \textbf{ARCH} (Autoregressive conditional heteroskedasticity) and
  \textbf{GARCH} (Generalized autoregressive conditional
  heteroskedasticity):
  https://en.wikipedia.org/wiki/Autoregressive\_conditional\_heteroskedasticity
  and https://arch.readthedocs.io/en/latest/index.html
\item
  \textbf{ARIMAX} (ARIMA model with regressors/predictors):
  http://www.statsmodels.org/dev/generated/statsmodels.tsa.arima\_model.ARIMA.html
  and https://pyflux.readthedocs.io/en/latest/arimax.html
\end{itemize}

    \paragraph{2.3.7 (Optional) Ensemble Method of Time Series
Models}\label{optional-ensemble-method-of-time-series-models}

\paragraph{To improve prediction accuracy, you can use ensemble method
of multiple time series
models}\label{to-improve-prediction-accuracy-you-can-use-ensemble-method-of-multiple-time-series-models}

References: * https://arxiv.org/ftp/arxiv/papers/1302/1302.6595.pdf *
https://ieeexplore.ieee.org/document/6011011

    \subsection{3 Recap: A typical procedure of time series
forecasting}\label{recap-a-typical-procedure-of-time-series-forecasting}

\begin{figure}
\centering
\includegraphics{procedure.jpg}
\caption{title}
\end{figure}

    \subsection{4 Assignments (5 points)}\label{assignments-5-points}

    \textbf{Purpose: Learn how to do simple analysis on time series data}

    \begin{Verbatim}[commandchars=\\\{\}]
{\color{incolor}In [{\color{incolor}23}]:} \PY{c+c1}{\PYZsh{} Import libraries}
         \PY{k+kn}{import} \PY{n+nn}{numpy} \PY{k}{as} \PY{n+nn}{np}
         \PY{k+kn}{import} \PY{n+nn}{pandas} \PY{k}{as} \PY{n+nn}{pd}
         \PY{k+kn}{import} \PY{n+nn}{matplotlib}\PY{n+nn}{.}\PY{n+nn}{pyplot} \PY{k}{as} \PY{n+nn}{plt}
         \PY{k+kn}{from} \PY{n+nn}{math} \PY{k}{import} \PY{n}{sqrt}
         \PY{k+kn}{import} \PY{n+nn}{statsmodels}\PY{n+nn}{.}\PY{n+nn}{api} \PY{k}{as} \PY{n+nn}{sm}
         \PY{k+kn}{from} \PY{n+nn}{IPython}\PY{n+nn}{.}\PY{n+nn}{display} \PY{k}{import} \PY{n}{Image}
         \PY{k+kn}{from} \PY{n+nn}{matplotlib}\PY{n+nn}{.}\PY{n+nn}{pylab} \PY{k}{import} \PY{n}{rcParams}
         \PY{k+kn}{from} \PY{n+nn}{statsmodels}\PY{n+nn}{.}\PY{n+nn}{tsa}\PY{n+nn}{.}\PY{n+nn}{stattools} \PY{k}{import} \PY{n}{adfuller}
         \PY{k+kn}{from} \PY{n+nn}{statsmodels}\PY{n+nn}{.}\PY{n+nn}{tsa}\PY{n+nn}{.}\PY{n+nn}{stattools} \PY{k}{import} \PY{n}{acf}\PY{p}{,} \PY{n}{pacf}
         \PY{k+kn}{from} \PY{n+nn}{sklearn}\PY{n+nn}{.}\PY{n+nn}{linear\PYZus{}model} \PY{k}{import} \PY{n}{LinearRegression}
         \PY{k+kn}{from} \PY{n+nn}{statsmodels}\PY{n+nn}{.}\PY{n+nn}{tsa}\PY{n+nn}{.}\PY{n+nn}{arima\PYZus{}model} \PY{k}{import} \PY{n}{ARMA}\PY{p}{,} \PY{n}{ARIMA}
         \PY{k+kn}{from} \PY{n+nn}{sklearn}\PY{n+nn}{.}\PY{n+nn}{metrics} \PY{k}{import} \PY{n}{explained\PYZus{}variance\PYZus{}score}
         \PY{k+kn}{from} \PY{n+nn}{\PYZus{}\PYZus{}future\PYZus{}\PYZus{}} \PY{k}{import} \PY{n}{division}\PY{p}{,} \PY{n}{print\PYZus{}function}
         \PY{k+kn}{from} \PY{n+nn}{datetime} \PY{k}{import} \PY{n}{datetime}
         \PY{o}{\PYZpc{}}\PY{k}{matplotlib} inline
\end{Verbatim}


    \paragraph{Dataset:}\label{dataset}

The Google Inc. stock price can be obtained from yahoo finance
https://sg.finance.yahoo.com/quote/GOOG/history?p=GOOG.

    \begin{Verbatim}[commandchars=\\\{\}]
{\color{incolor}In [{\color{incolor}24}]:} \PY{c+c1}{\PYZsh{} Load dataset: You need to download Google stock price from yahoo finance}
         \PY{o}{\PYZpc{}}\PY{k}{pwd}
         \PY{n}{google} \PY{o}{=} \PY{n}{pd}\PY{o}{.}\PY{n}{read\PYZus{}csv}\PY{p}{(}\PY{l+s+s1}{\PYZsq{}}\PY{l+s+s1}{./GOOG.csv}\PY{l+s+s1}{\PYZsq{}}\PY{p}{)}\PY{p}{[}\PY{p}{[}\PY{l+s+s1}{\PYZsq{}}\PY{l+s+s1}{Date}\PY{l+s+s1}{\PYZsq{}}\PY{p}{,}\PY{l+s+s1}{\PYZsq{}}\PY{l+s+s1}{Adj Close}\PY{l+s+s1}{\PYZsq{}}\PY{p}{]}\PY{p}{]}
         
         \PY{c+c1}{\PYZsh{} Transform Date column to date type}
         \PY{n}{google}\PY{p}{[}\PY{l+s+s1}{\PYZsq{}}\PY{l+s+s1}{Date}\PY{l+s+s1}{\PYZsq{}}\PY{p}{]} \PY{o}{=} \PY{n}{google}\PY{p}{[}\PY{l+s+s1}{\PYZsq{}}\PY{l+s+s1}{Date}\PY{l+s+s1}{\PYZsq{}}\PY{p}{]}\PY{o}{.}\PY{n}{apply}\PY{p}{(}\PY{n+nb}{str}\PY{p}{)}
         \PY{n}{google}\PY{p}{[}\PY{l+s+s1}{\PYZsq{}}\PY{l+s+s1}{Date}\PY{l+s+s1}{\PYZsq{}}\PY{p}{]} \PY{o}{=} \PY{n}{pd}\PY{o}{.}\PY{n}{to\PYZus{}datetime}\PY{p}{(}\PY{n}{google}\PY{p}{[}\PY{l+s+s1}{\PYZsq{}}\PY{l+s+s1}{Date}\PY{l+s+s1}{\PYZsq{}}\PY{p}{]}\PY{p}{,} \PY{n}{infer\PYZus{}datetime\PYZus{}format}\PY{o}{=}\PY{k+kc}{True}\PY{p}{)}
         \PY{n}{google}\PY{o}{.}\PY{n}{head}\PY{p}{(}\PY{n}{n}\PY{o}{=}\PY{l+m+mi}{10}\PY{p}{)}
\end{Verbatim}


\begin{Verbatim}[commandchars=\\\{\}]
{\color{outcolor}Out[{\color{outcolor}24}]:}         Date   Adj Close
         0 2017-01-03  786.140015
         1 2017-01-04  786.900024
         2 2017-01-05  794.020020
         3 2017-01-06  806.150024
         4 2017-01-09  806.650024
         5 2017-01-10  804.789978
         6 2017-01-11  807.909973
         7 2017-01-12  806.359985
         8 2017-01-13  807.880005
         9 2017-01-17  804.609985
\end{Verbatim}
            
    \begin{Verbatim}[commandchars=\\\{\}]
{\color{incolor}In [{\color{incolor}25}]:} \PY{c+c1}{\PYZsh{} Split the dataset: First 70\PYZpc{} for training, Last 30\PYZpc{} for testing}
         \PY{n}{google\PYZus{}size} \PY{o}{=} \PY{n}{google}\PY{o}{.}\PY{n}{shape}\PY{p}{[}\PY{l+m+mi}{0}\PY{p}{]}
         \PY{n}{google\PYZus{}train} \PY{o}{=} \PY{n}{google}\PY{o}{.}\PY{n}{loc}\PY{p}{[}\PY{p}{:}\PY{n}{google\PYZus{}size}\PY{o}{*}\PY{l+m+mf}{0.7}\PY{p}{]}
         \PY{n}{google\PYZus{}test} \PY{o}{=} \PY{n}{google}\PY{o}{.}\PY{n}{loc}\PY{p}{[}\PY{n}{google\PYZus{}size}\PY{o}{*}\PY{l+m+mf}{0.7}\PY{p}{:}\PY{p}{]}
\end{Verbatim}


    \begin{Verbatim}[commandchars=\\\{\}]
{\color{incolor}In [{\color{incolor}26}]:} \PY{c+c1}{\PYZsh{} Plotting Adjusted Closing Price of Google Stock on a Time Series}
         \PY{n}{plt}\PY{o}{.}\PY{n}{figure}\PY{p}{(}\PY{n}{figsize}\PY{o}{=}\PY{p}{(}\PY{l+m+mi}{15}\PY{p}{,}\PY{l+m+mi}{5}\PY{p}{)}\PY{p}{)}
         \PY{n}{plt}\PY{o}{.}\PY{n}{grid}\PY{p}{(}\PY{k+kc}{True}\PY{p}{)}
         \PY{n}{plt}\PY{o}{.}\PY{n}{plot}\PY{p}{(}\PY{n}{google}\PY{p}{[}\PY{l+s+s1}{\PYZsq{}}\PY{l+s+s1}{Date}\PY{l+s+s1}{\PYZsq{}}\PY{p}{]}\PY{p}{,} \PY{n}{google}\PY{p}{[}\PY{l+s+s1}{\PYZsq{}}\PY{l+s+s1}{Adj Close}\PY{l+s+s1}{\PYZsq{}}\PY{p}{]}\PY{p}{)}
         \PY{n}{plt}\PY{o}{.}\PY{n}{xlabel}\PY{p}{(}\PY{l+s+s1}{\PYZsq{}}\PY{l+s+s1}{Date}\PY{l+s+s1}{\PYZsq{}}\PY{p}{)}
         \PY{n}{plt}\PY{o}{.}\PY{n}{ylabel}\PY{p}{(}\PY{l+s+s1}{\PYZsq{}}\PY{l+s+s1}{Adjusted Closing Index}\PY{l+s+s1}{\PYZsq{}}\PY{p}{)}
         \PY{n}{plt}\PY{o}{.}\PY{n}{title}\PY{p}{(}\PY{l+s+s1}{\PYZsq{}}\PY{l+s+s1}{Google Inc. Adjusted Closing Price \PYZhy{} Year 2017}\PY{l+s+s1}{\PYZsq{}}\PY{p}{)}
         \PY{n}{locs}\PY{p}{,} \PY{n}{labels} \PY{o}{=} \PY{n}{plt}\PY{o}{.}\PY{n}{xticks}\PY{p}{(}\PY{p}{)}
         \PY{n}{plt}\PY{o}{.}\PY{n}{setp}\PY{p}{(}\PY{n}{labels}\PY{p}{,} \PY{n}{rotation}\PY{o}{=}\PY{l+m+mi}{45}\PY{p}{)}
         \PY{n}{plt}\PY{o}{.}\PY{n}{show}\PY{p}{(}\PY{p}{)}
\end{Verbatim}


    \begin{center}
    \adjustimage{max size={0.9\linewidth}{0.9\paperheight}}{output_57_0.png}
    \end{center}
    { \hspace*{\fill} \\}
    
    \textbf{Question 1. Do you think this is a stationary time series? (1
point)}

Hint: * Dickey-Fuller Test on \texttt{google\_train} training data

\begin{itemize}
\tightlist
\item
  Sample code:
\end{itemize}

\begin{Shaded}
\begin{Highlighting}[]
\CommentTok{# Dicky-Fuller test on stationary condition}
\BuiltInTok{print}\NormalTok{ (}\StringTok{'Results of Dickey-Fuller Test:'}\NormalTok{)}

\CommentTok{# Dickey-Fuller test}
\NormalTok{dftest }\OperatorTok{=}\NormalTok{ adfuller(data_train[}\StringTok{'Adj Close'}\NormalTok{], autolag}\OperatorTok{=}\VariableTok{None}\NormalTok{) }
\NormalTok{dfoutput }\OperatorTok{=}\NormalTok{ pd.Series(dftest[}\DecValTok{0}\NormalTok{:}\DecValTok{4}\NormalTok{], index}\OperatorTok{=}\NormalTok{[}\StringTok{'Test Statistic'}\NormalTok{,}\StringTok{'p-value'}\NormalTok{,}\StringTok{'#Lags Used'}\NormalTok{,}\StringTok{'Number of Observations Used'}\NormalTok{])}

\ControlFlowTok{for}\NormalTok{ key, value }\KeywordTok{in}\NormalTok{ dftest[}\DecValTok{4}\NormalTok{].items():}
\NormalTok{    dfoutput[}\StringTok{'Critical Value (}\SpecialCharTok\NormalTok{key] }\OperatorTok{=}\NormalTok{ value}

\BuiltInTok{print}\NormalTok{ (dfoutput)}
\end{Highlighting}
\end{Shaded}

\begin{itemize}
\tightlist
\item
  \textbf{Note: Please interpret your results.}
\end{itemize}

    \begin{Verbatim}[commandchars=\\\{\}]
{\color{incolor}In [{\color{incolor} }]:} \PY{c+c1}{\PYZsh{}\PYZsh{}\PYZsh{}\PYZsh{} Your interpretation of Dickey\PYZhy{}Fuller test}
        
        \PY{n}{Answer}\PY{p}{:} 
\end{Verbatim}


    \textbf{Question 2. De-trend time series data using first differencing
method, and Redo Dickey-Fuller test (2 points)}

Hint: * First differencing on \texttt{google\_train} * Dickey-Fuller
test on first-differenced data * Sample code:

\begin{Shaded}
\begin{Highlighting}[]
\CommentTok{# Applying First-Order Differencing to the original time series data}
\NormalTok{data_diff }\OperatorTok{=}\NormalTok{ data_train[}\StringTok{'Adj Close'}\NormalTok{] }\OperatorTok{-}\NormalTok{ data_train[}\StringTok{'Adj Close'}\NormalTok{].shift(}\DecValTok{1}\NormalTok{)}
\NormalTok{data_diff.dropna(inplace}\OperatorTok{=}\VariableTok{True}\NormalTok{)}

\CommentTok{# Dicky-Fuller test on stationary condition}
\BuiltInTok{print}\NormalTok{ (}\StringTok{'Results of Dickey-Fuller Test:'}\NormalTok{)}

\CommentTok{# Dickey-Fuller test}
\NormalTok{dftest }\OperatorTok{=}\NormalTok{ adfuller(data_diff, autolag}\OperatorTok{=}\VariableTok{None}\NormalTok{) }
\NormalTok{dfoutput }\OperatorTok{=}\NormalTok{ pd.Series(dftest[}\DecValTok{0}\NormalTok{:}\DecValTok{4}\NormalTok{], index}\OperatorTok{=}\NormalTok{[}\StringTok{'Test Statistic'}\NormalTok{,}\StringTok{'p-value'}\NormalTok{,}\StringTok{'#Lags Used'}\NormalTok{,}\StringTok{'Number of Observations Used'}\NormalTok{])}

\ControlFlowTok{for}\NormalTok{ key, value }\KeywordTok{in}\NormalTok{ dftest[}\DecValTok{4}\NormalTok{].items():}
\NormalTok{    dfoutput[}\StringTok{'Critical Value (}\SpecialCharTok\NormalTok{key] }\OperatorTok{=}\NormalTok{ value}

\BuiltInTok{print}\NormalTok{ (dfoutput)}
\end{Highlighting}
\end{Shaded}

\begin{itemize}
\tightlist
\item
  \textbf{Note: Please interpret your results.}
\end{itemize}

    \begin{Verbatim}[commandchars=\\\{\}]
{\color{incolor}In [{\color{incolor} }]:} \PY{c+c1}{\PYZsh{}\PYZsh{}\PYZsh{}\PYZsh{} Your interpretation of Dickey\PYZhy{}Fuller test after doing first\PYZhy{}differencing on the original time series data}
        
        \PY{n}{Answer}\PY{p}{:} 
\end{Verbatim}


    \textbf{Question 3. Run an ARIMA(2,1,1) model on original time series
data, and forecast what is the next value (2 points)}

Hint: * ARIMA(2,1,1): AR(2), MA(1) and first-differencing; p=2, q=1, d=1
for ARIMA(p,d,q) * Sample code:

\begin{Shaded}
\begin{Highlighting}[]
\CommentTok{# Suppose we use ARIMA(5,1,2) to fit the log-transformed time series data: AR(5), MA(2) and first-order-differencing}
\CommentTok{# Remember: we have tested that first-order-differencing is effective enough to remove trending. }
\ImportTok{from}\NormalTok{ statsmodels.tsa.arima_model }\ImportTok{import}\NormalTok{ ARIMA}
\ImportTok{import}\NormalTok{ seaborn }\ImportTok{as}\NormalTok{ sns}

\CommentTok{# AR(5), MA(2) and first-order-differencing}
\NormalTok{model }\OperatorTok{=}\NormalTok{ ARIMA(data_train[}\StringTok{'Adj Close'}\NormalTok{], order}\OperatorTok{=}\NormalTok{(}\DecValTok{5}\NormalTok{,}\DecValTok{1}\NormalTok{,}\DecValTok{2}\NormalTok{))}
\NormalTok{model_fit }\OperatorTok{=}\NormalTok{ model.fit(disp}\OperatorTok{=}\DecValTok{0}\NormalTok{)}

\CommentTok{# Print model results}
\BuiltInTok{print}\NormalTok{ (model_fit.summary()) }

\CommentTok{# Find AIC value}
\BuiltInTok{print}\NormalTok{ (}\StringTok{"AIC is: }\SpecialCharTok\NormalTok{(model_fit.aic))}

\CommentTok{# Predict/Forecast what is the next value}
\NormalTok{yhat }\OperatorTok{=}\NormalTok{ model_fit.forecast()[}\DecValTok{0}\NormalTok{][}\DecValTok{0}\NormalTok{] }\CommentTok{# Forecast the next instance}
\BuiltInTok{print}\NormalTok{ (}\StringTok{"Predicted next value: }\SpecialCharTok\NormalTok{(yhat))}
\end{Highlighting}
\end{Shaded}

\begin{itemize}
\tightlist
\item
  \textbf{Note 1: Please print out your ARIMA model results (with AIC
  value).}
\item
  \textbf{Note 2: Please print out your predicted value of next time
  period.}
\end{itemize}

References:
https://www.statsmodels.org/dev/generated/statsmodels.tsa.arima\_model.ARIMA.html

    \subsection{5 References}\label{references}

{[}1{]} Haroon, D. Python Machine Learning Case Studies. Apress..
{[}2{]} statsmodels module: https://www.statsmodels.org/dev/tsa.html


    % Add a bibliography block to the postdoc
    
    
    
    \end{document}
